\documentclass{article}
\usepackage{graphicx}
\usepackage{amsmath}
\usepackage{array}
\usepackage{fancyhdr}
\usepackage{amssymb}
\usepackage[shortlabels]{enumitem}

\DeclareMathOperator{\R}{\mathbb R}

\pagestyle{fancy}
\fancyhead[L]{Banghao Chi}
\fancyhead[C]{Homework 5}
\fancyhead[R]{27th Mar}

\fancyfoot[C]{\thepage}

\renewcommand{\headrulewidth}{0.5pt}
\renewcommand{\footrulewidth}{0.5pt}

\begin{document}

\section*{Exercise 1}
\begin{enumerate}
\item[(a)] Let
$$A = \begin{pmatrix} 1 & 3 \\ 2 & -1 \end{pmatrix}, \quad B = \begin{pmatrix} 1 & 0 & -3 \\ 4 & 1 & 2 \end{pmatrix}, \quad C = \begin{pmatrix} 1 & 1 & 4 \\ -1 & -2 & 0 \end{pmatrix}, \text{ and } \quad D = \begin{pmatrix} 2 \\ -2 \\ 3 \end{pmatrix}.$$
Compute $A(2B + 3C)$, $(AB)D$, and $A(BD)$.
\item[(b)] Let
$$A = \begin{pmatrix} 2 & 5 \\ -3 & 1 \\ 4 & 2 \end{pmatrix}, \quad B = \begin{pmatrix} 3 & -2 & 0 \\ 1 & -1 & 4 \\ 5 & 5 & 3 \end{pmatrix}, \text{ and } \quad C = \begin{pmatrix} 4 & 0 & 3 \end{pmatrix}.$$
Compute $A^t$, $A^tB$, $BC^t$, $CB$, and $CA$.
\end{enumerate}

\textbf{Solution: } \\

\begin{enumerate}
\item[(a)] 

\begin{enumerate}
\item[1.] $2B + 3C$:
\begin{align*}
2B &= \begin{pmatrix} 2 & 0 & -6 \\ 8 & 2 & 4 \end{pmatrix} \\
3C &= \begin{pmatrix} 3 & 3 & 12 \\ -3 & -6 & 0 \end{pmatrix} \\
2B + 3C &= \begin{pmatrix} 5 & 3 & 6 \\ 5 & -4 & 4 \end{pmatrix}
\end{align*}

Therefore, $A(2B + 3C)$:
\begin{align*}
A(2B + 3C) &= \begin{pmatrix} 1 & 3 \\ 2 & -1 \end{pmatrix} \begin{pmatrix} 5 & 3 & 6 \\ 5 & -4 & 4 \end{pmatrix} \\
&= \begin{pmatrix} 20 & -9 & 18 \\ 5 & 10 & 8 \end{pmatrix}
\end{align*}

\item[2.] $AB$:
\begin{align*}
AB &= \begin{pmatrix} 1 & 3 \\ 2 & -1 \end{pmatrix} \begin{pmatrix} 1 & 0 & -3 \\ 4 & 1 & 2 \end{pmatrix} \\
&= \begin{pmatrix} 13 & 3 & 3 \\ -2 & -1 & -8 \end{pmatrix}
\end{align*}

Therefore, $(AB)D$:
\begin{align*}
(AB)D &= \begin{pmatrix} 13 & 3 & 3 \\ -2 & -1 & -8 \end{pmatrix} \begin{pmatrix} 2 \\ -2 \\ 3 \end{pmatrix} \\
&= \begin{pmatrix} 29 \\ -26 \end{pmatrix}
\end{align*}

\item[3.] $BD$:
\begin{align*}
BD &= \begin{pmatrix} 1 & 0 & -3 \\ 4 & 1 & 2 \end{pmatrix} \begin{pmatrix} 2 \\ -2 \\ 3 \end{pmatrix} \\
&= \begin{pmatrix} -7 \\ 12 \end{pmatrix}
\end{align*}

Therefore, $A(BD)$:
\begin{align*}
A(BD) &= \begin{pmatrix} 1 & 3 \\ 2 & -1 \end{pmatrix} \begin{pmatrix} -7 \\ 12 \end{pmatrix} \\
&= \begin{pmatrix} 29 \\ -26 \end{pmatrix}
\end{align*}
\end{enumerate}

\item[(b)]

\begin{enumerate}
\item[1.] $A^t$:
\begin{align*}
A^t = \begin{pmatrix} 2 & -3 & 4 \\ 5 & 1 & 2 \end{pmatrix}
\end{align*}

\item[2.] $A^tB$:
\begin{align*}
A^tB &= \begin{pmatrix} 2 & -3 & 4 \\ 5 & 1 & 2 \end{pmatrix} \begin{pmatrix} 3 & -2 & 0 \\ 1 & -1 & 4 \\ 5 & 5 & 3 \end{pmatrix} \\
&= \begin{pmatrix} 23 & 19 & 0 \\ 26 & -1 & 10 \end{pmatrix}
\end{align*}

\item[3.] $C^t$:
\begin{align*}
C^t = \begin{pmatrix} 4 \\ 0 \\ 3 \end{pmatrix}
\end{align*}

Therefore, $BC^t$:
\begin{align*}
BC^t &= \begin{pmatrix} 3 & -2 & 0 \\ 1 & -1 & 4 \\ 5 & 5 & 3 \end{pmatrix} \begin{pmatrix} 4 \\ 0 \\ 3 \end{pmatrix} \\
&= \begin{pmatrix} 12 \\ 16 \\ 29 \end{pmatrix}
\end{align*}

\item[4.] $CB$:
\begin{align*}
CB &= \begin{pmatrix} 4 & 0 & 3 \end{pmatrix} \begin{pmatrix} 3 & -2 & 0 \\ 1 & -1 & 4 \\ 5 & 5 & 3 \end{pmatrix} \\
&= \begin{pmatrix} 27 & 7 & 9 \end{pmatrix}
\end{align*}

\item[5.] $CA$:
\begin{align*}
CA &= \begin{pmatrix} 4 & 0 & 3 \end{pmatrix} \begin{pmatrix} 2 & 5 \\ -3 & 1 \\ 4 & 2 \end{pmatrix} \\
&= \begin{pmatrix} 20 & 26 \end{pmatrix}
\end{align*}
\end{enumerate}
\end{enumerate}

\newpage

\section*{Exercise 2}
Find two matrices $A, B \in \mathcal{M}_{2\times 2}(\mathbb{R})$ where $AB$ is the zero matrix but $BA$ is not.

\textbf{Solution: } \\



\newpage

\section*{Exercise 3}
Let $g(x) = 3 + x$. Let $T : \mathcal{P}_2(\mathbb{R}) \to \mathcal{P}_2(\mathbb{R})$ and $U : \mathcal{P}_2(\mathbb{R}) \to \mathbb{R}^3$ be the linear transformations respectively defined by
$$T(f(x)) = f'(x)g(x) + 2f(x) \text{ and } U(a + bx + cx^2) = (a + b, c, a - b).$$

Let $\beta$ and $\gamma$ be the standard ordered bases of $\mathcal{P}_2(\mathbb{R})$ and $\mathbb{R}^3$, respectively.

\begin{enumerate}
\item[(a)] Compute $[U]_{\beta}^{\gamma}$, $[T]_{\beta}$, and $[UT]_{\beta}^{\gamma}$ directly. Then use $[UT]_{\beta}^{\gamma} = [U]_{\beta}^{\gamma}[T]_{\beta}$ to verify your result.
\item[(b)] Let $h(x) = 3 - 2x + x^2$. Compute $[h]_{\beta}$ and $[U(h)]_{\gamma}$. Then use $[U]_{\beta}^{\gamma}$ from (a) and $[U(h)]_{\gamma} = [U]_{\beta}^{\gamma}[h]_{\beta}$ to verify your result.
\end{enumerate}

\textbf{Solution: } \\



\newpage

\section*{Exercise 4}
Suppose $A$ and $B$ are invertible $n \times n$ matrices.

\begin{enumerate}
\item[(a)] Prove that $(AB)^{-1} = B^{-1}A^{-1}$.
\item[(b)] Prove that $(A^t)^{-1} = (A^{-1})^t$.
\end{enumerate}

\textbf{Solution: } \\



\newpage

\section*{Exercise 5}
\begin{enumerate}
\item[(a)] Let $A$ and $B$ be $n \times n$ matrices such that $AB$ is invertible. Prove that both $A$ and $B$ are invertible.
\item[(b)] Give an example of two noninvertible matrices whose product is invertible.
\item[(c)] Prove or give a counterexample: If $A$ and $B$ are nonzero $n \times n$ matrices with $AB$ the zero matrix then $A$ is not invertible.
\end{enumerate}

\textbf{Solution: } \\



\newpage

\section*{Exercise 6}
Find the inverse of the following matrix, and check your answer two different ways.
$$A = \begin{pmatrix} 2 & 1 & 2 \\ 1 & -1 & 0 \\ 4 & -2 & 1 \end{pmatrix}.$$

\textbf{Solution: } \\



\end{document}