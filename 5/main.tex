\documentclass{article}
\usepackage{graphicx}
\usepackage{amsmath}
\usepackage{array}
\usepackage{fancyhdr}
\usepackage{amssymb}
\usepackage[shortlabels]{enumitem}

\DeclareMathOperator{\R}{\mathbb R}

\pagestyle{fancy}
\fancyhead[L]{Banghao Chi}
\fancyhead[C]{Homework 5}
\fancyhead[R]{27th Mar}

\fancyfoot[C]{\thepage}

\renewcommand{\headrulewidth}{0.5pt}
\renewcommand{\footrulewidth}{0.5pt}

\begin{document}

\section*{Exercise 1}
\begin{enumerate}
\item[(a)] Let
$$A = \begin{pmatrix} 1 & 3 \\ 2 & -1 \end{pmatrix}, \quad B = \begin{pmatrix} 1 & 0 & -3 \\ 4 & 1 & 2 \end{pmatrix}, \quad C = \begin{pmatrix} 1 & 1 & 4 \\ -1 & -2 & 0 \end{pmatrix}, \text{ and } \quad D = \begin{pmatrix} 2 \\ -2 \\ 3 \end{pmatrix}.$$
Compute $A(2B + 3C)$, $(AB)D$, and $A(BD)$.
\item[(b)] Let
$$A = \begin{pmatrix} 2 & 5 \\ -3 & 1 \\ 4 & 2 \end{pmatrix}, \quad B = \begin{pmatrix} 3 & -2 & 0 \\ 1 & -1 & 4 \\ 5 & 5 & 3 \end{pmatrix}, \text{ and } \quad C = \begin{pmatrix} 4 & 0 & 3 \end{pmatrix}.$$
Compute $A^t$, $A^tB$, $BC^t$, $CB$, and $CA$.
\end{enumerate}

\textbf{Solution: } \\

\begin{enumerate}
\item[(a)] 

\begin{enumerate}
\item[1.] $2B + 3C$:
\begin{align*}
2B &= \begin{pmatrix} 2 & 0 & -6 \\ 8 & 2 & 4 \end{pmatrix} \\
3C &= \begin{pmatrix} 3 & 3 & 12 \\ -3 & -6 & 0 \end{pmatrix} \\
2B + 3C &= \begin{pmatrix} 5 & 3 & 6 \\ 5 & -4 & 4 \end{pmatrix}
\end{align*}

Therefore, $A(2B + 3C)$:
\begin{align*}
A(2B + 3C) &= \begin{pmatrix} 1 & 3 \\ 2 & -1 \end{pmatrix} \begin{pmatrix} 5 & 3 & 6 \\ 5 & -4 & 4 \end{pmatrix} \\
&= \begin{pmatrix} 20 & -9 & 18 \\ 5 & 10 & 8 \end{pmatrix}
\end{align*}

\item[2.] $AB$:
\begin{align*}
AB &= \begin{pmatrix} 1 & 3 \\ 2 & -1 \end{pmatrix} \begin{pmatrix} 1 & 0 & -3 \\ 4 & 1 & 2 \end{pmatrix} \\
&= \begin{pmatrix} 13 & 3 & 3 \\ -2 & -1 & -8 \end{pmatrix}
\end{align*}

Therefore, $(AB)D$:
\begin{align*}
(AB)D &= \begin{pmatrix} 13 & 3 & 3 \\ -2 & -1 & -8 \end{pmatrix} \begin{pmatrix} 2 \\ -2 \\ 3 \end{pmatrix} \\
&= \begin{pmatrix} 29 \\ -26 \end{pmatrix}
\end{align*}

\item[3.] $BD$:
\begin{align*}
BD &= \begin{pmatrix} 1 & 0 & -3 \\ 4 & 1 & 2 \end{pmatrix} \begin{pmatrix} 2 \\ -2 \\ 3 \end{pmatrix} \\
&= \begin{pmatrix} -7 \\ 12 \end{pmatrix}
\end{align*}

Therefore, $A(BD)$:
\begin{align*}
A(BD) &= \begin{pmatrix} 1 & 3 \\ 2 & -1 \end{pmatrix} \begin{pmatrix} -7 \\ 12 \end{pmatrix} \\
&= \begin{pmatrix} 29 \\ -26 \end{pmatrix}
\end{align*}
\end{enumerate}

\item[(b)]

\begin{enumerate}
\item[1.] $A^t$:
\begin{align*}
A^t = \begin{pmatrix} 2 & -3 & 4 \\ 5 & 1 & 2 \end{pmatrix}
\end{align*}

\item[2.] $A^tB$:
\begin{align*}
A^tB &= \begin{pmatrix} 2 & -3 & 4 \\ 5 & 1 & 2 \end{pmatrix} \begin{pmatrix} 3 & -2 & 0 \\ 1 & -1 & 4 \\ 5 & 5 & 3 \end{pmatrix} \\
&= \begin{pmatrix} 23 & 19 & 0 \\ 26 & -1 & 10 \end{pmatrix}
\end{align*}

\item[3.] $C^t$:
\begin{align*}
C^t = \begin{pmatrix} 4 \\ 0 \\ 3 \end{pmatrix}
\end{align*}

Therefore, $BC^t$:
\begin{align*}
BC^t &= \begin{pmatrix} 3 & -2 & 0 \\ 1 & -1 & 4 \\ 5 & 5 & 3 \end{pmatrix} \begin{pmatrix} 4 \\ 0 \\ 3 \end{pmatrix} \\
&= \begin{pmatrix} 12 \\ 16 \\ 29 \end{pmatrix}
\end{align*}

\item[4.] $CB$:
\begin{align*}
CB &= \begin{pmatrix} 4 & 0 & 3 \end{pmatrix} \begin{pmatrix} 3 & -2 & 0 \\ 1 & -1 & 4 \\ 5 & 5 & 3 \end{pmatrix} \\
&= \begin{pmatrix} 27 & 7 & 9 \end{pmatrix}
\end{align*}

\item[5.] $CA$:
\begin{align*}
CA &= \begin{pmatrix} 4 & 0 & 3 \end{pmatrix} \begin{pmatrix} 2 & 5 \\ -3 & 1 \\ 4 & 2 \end{pmatrix} \\
&= \begin{pmatrix} 20 & 26 \end{pmatrix}
\end{align*}
\end{enumerate}
\end{enumerate}

\newpage

\section*{Exercise 2}
Find two matrices $A, B \in \mathcal{M}_{2\times 2}(\mathbb{R})$ where $AB$ is the zero matrix but $BA$ is not. \\

\textbf{Solution: } \\

Let $A = \begin{pmatrix} 1 & 0 \\ 0 & 0 \end{pmatrix}$ and $B = \begin{pmatrix} 0 & 0 \\ 1 & 0 \end{pmatrix}$. \\

We have the product $AB$:
\begin{align*}
AB &= \begin{pmatrix} 1 & 0 \\ 0 & 0 \end{pmatrix} \begin{pmatrix} 0 & 0 \\ 1 & 0 \end{pmatrix} \\
&= \begin{pmatrix} 0 & 0 \\ 0 & 0 \end{pmatrix}
\end{align*}

But for the product $BA$:
\begin{align*}
BA &= \begin{pmatrix} 0 & 0 \\ 1 & 0 \end{pmatrix} \begin{pmatrix} 1 & 0 \\ 0 & 0 \end{pmatrix} \\
&= \begin{pmatrix} 0 & 0 \\ 1 & 0 \end{pmatrix} \neq \begin{pmatrix} 0 & 0 \\ 0 & 0 \end{pmatrix}
\end{align*}

\newpage

\section*{Exercise 3}
Let $g(x) = 3 + x$. Let $T : \mathcal{P}_2(\mathbb{R}) \to \mathcal{P}_2(\mathbb{R})$ and $U : \mathcal{P}_2(\mathbb{R}) \to \mathbb{R}^3$ be the linear transformations respectively defined by
$$T(f(x)) = f'(x)g(x) + 2f(x) \text{ and } U(a + bx + cx^2) = (a + b, c, a - b).$$

Let $\beta$ and $\gamma$ be the standard ordered bases of $\mathcal{P}_2(\mathbb{R})$ and $\mathbb{R}^3$, respectively.

\begin{enumerate}
\item[(a)] Compute $[U]_{\beta}^{\gamma}$, $[T]_{\beta}$, and $[UT]_{\beta}^{\gamma}$ directly. Then use $[UT]_{\beta}^{\gamma} = [U]_{\beta}^{\gamma}[T]_{\beta}$ to verify your result.
\item[(b)] Let $h(x) = 3 - 2x + x^2$. Compute $[h]_{\beta}$ and $[U(h)]_{\gamma}$. Then use $[U]_{\beta}^{\gamma}$ from (a) and $[U(h)]_{\gamma} = [U]_{\beta}^{\gamma}[h]_{\beta}$ to verify your result.
\end{enumerate}

\textbf{Solution: } \\

\begin{enumerate}
\item[(a)] 

\begin{enumerate}
\item[1.] $[U]_{\beta}^{\gamma}$:
\begin{align*}
U(1) &= (1+0, 0, 1-0) = (1, 0, 1) \\
U(x) &= (0+1, 0, 0-1) = (1, 0, -1) \\
U(x^2) &= (0+0, 1, 0-0) = (0, 1, 0)
\end{align*}

So, $[U]_{\beta}^{\gamma}$ has these vectors as columns:
$$[U]_{\beta}^{\gamma} = 
\begin{pmatrix}
1 & 1 & 0 \\
0 & 0 & 1 \\
1 & -1 & 0
\end{pmatrix}$$

\item[2.] $[T]_{\beta}$:
\begin{align*}
T(1) &= 0 \cdot (3+x) + 2 \cdot 1 = 2 = 2 \cdot 1 + 0 \cdot x + 0 \cdot x^2 \\
T(x) &= 1 \cdot (3+x) + 2x = 3 + 3x = 3 \cdot 1 + 3 \cdot x + 0 \cdot x^2 \\
T(x^2) &= 2x \cdot (3+x) + 2x^2 = 6x + 2x^2 + 2x^2 = 0 \cdot 1 + 6 \cdot x + 4 \cdot x^2
\end{align*}

So, $[T]_{\beta}$ has these vectors as columns:
$$[T]_{\beta} = 
\begin{pmatrix}
2 & 3 & 0 \\
0 & 3 & 6 \\
0 & 0 & 4
\end{pmatrix}$$

\item[3.] $[UT]_{\beta}^{\gamma}$:
\begin{align*}
U(T(1)) &= U(2) = (2+0, 0, 2-0) = (2, 0, 2) \\
U(T(x)) &= U(3+3x) = (3+3, 0, 3-3) = (6, 0, 0) \\
U(T(x^2)) &= U(6x+4x^2) = (0+6, 4, 0-6) = (6, 4, -6)
\end{align*}

So, $[UT]_{\beta}^{\gamma}$ has these vectors as columns:
$$[UT]_{\beta}^{\gamma} = 
\begin{pmatrix}
2 & 6 & 6 \\
0 & 0 & 4 \\
2 & 0 & -6
\end{pmatrix}$$

\item[4.] Verify that $[UT]_{\beta}^{\gamma} = [U]_{\beta}^{\gamma}[T]_{\beta}$:
\begin{align*}
[U]_{\beta}^{\gamma}[T]_{\beta} &= 
\begin{pmatrix}
1 & 1 & 0 \\
0 & 0 & 1 \\
1 & -1 & 0
\end{pmatrix}
\begin{pmatrix}
2 & 3 & 0 \\
0 & 3 & 6 \\
0 & 0 & 4
\end{pmatrix} \\
&= 
\begin{pmatrix}
2 & 6 & 6 \\
0 & 0 & 4 \\
2 & 0 & -6
\end{pmatrix} = [UT]_{\beta}^{\gamma}
\end{align*}

\item[(b)]

\begin{enumerate}
\item[1.] $[h]_{\beta}$:
\begin{align*}
h(x) &= 3 - 2x + x^2 \\
&= 3 \cdot 1 + (-2) \cdot x + 1 \cdot x^2
\end{align*}

So, $[h]_{\beta}$ is the coordinate vector of $h$ with respect to $\beta$:
$$[h]_{\beta} = 
\begin{pmatrix}
3 \\
-2 \\
1
\end{pmatrix}$$

\item[2.] $[U(h)]_{\gamma}$:
\begin{align*}
U(h) &= U(3-2x+x^2) = (3+(-2), 1, 3-(-2)) = (1, 1, 5)
\end{align*}

So, $[U(h)]_{\gamma} = (1, 1, 5)^T$ \\

\item[3.] Verify that $[U(h)]_{\gamma} = [U]_{\beta}^{\gamma}[h]_{\beta}$:
\begin{align*}
[U]_{\beta}^{\gamma}[h]_{\beta} &= 
\begin{pmatrix}
1 & 1 & 0 \\
0 & 0 & 1 \\
1 & -1 & 0
\end{pmatrix}
\begin{pmatrix}
3 \\
-2 \\
1
\end{pmatrix} \\
&= 
\begin{pmatrix}
1 \\
1 \\
5
\end{pmatrix} = [U(h)]_{\gamma}
\end{align*}
\end{enumerate}
\end{enumerate}
\end{enumerate}

\newpage

\section*{Exercise 4}
Suppose $A$ and $B$ are invertible $n \times n$ matrices.

\begin{enumerate}
\item[(a)] Prove that $(AB)^{-1} = B^{-1}A^{-1}$.
\item[(b)] Prove that $(A^t)^{-1} = (A^{-1})^t$.
\end{enumerate}

\textbf{Solution: } \\

(a)
\begin{align*}
(AB)(B^{-1}A^{-1}) &= A(BB^{-1})A^{-1} \\
&= A(I)A^{-1} \\
&= AA^{-1} \\
&= I
\end{align*}

\begin{align*}
(B^{-1}A^{-1})(AB) &= B^{-1}(A^{-1}A)B \\
&= B^{-1}(I)B \\
&= B^{-1}B \\
&= I
\end{align*}

Since both products equal the identity matrix $I$, we have proven that $(AB)^{-1} = B^{-1}A^{-1}$. \\

(b)
Using the property that $(CD)^t = D^t C^t$ for any matrices $C$ and $D$:
\begin{align*}
(A^t)((A^{-1})^t) &= (A^{-1}A)^t \\
&= I^t \\
&= I
\end{align*}

Similarly:
\begin{align*}
((A^{-1})^t)(A^t) &= (AA^{-1})^t \\
&= I^t \\
&= I
\end{align*}

Since both products equal the identity matrix $I$, we have proven that $(A^t)^{-1} = (A^{-1})^t$.

\newpage

\section*{Exercise 5}
\begin{enumerate}
\item[(a)] Let $A$ and $B$ be $n \times n$ matrices such that $AB$ is invertible. Prove that both $A$ and $B$ are invertible.
\item[(b)] Give an example of two noninvertible matrices whose product is invertible.
\item[(c)] Prove or give a counterexample: If $A$ and $B$ are nonzero $n \times n$ matrices with $AB$ the zero matrix then $A$ is not invertible.
\end{enumerate}

\textbf{Solution: } \\

\begin{enumerate}
\item[(a)]

Since $AB$ is invertible, there exists a matrix $C$ such that $(AB)C = I$.

Define $D = BC$. Then:
\begin{align*}
AD &= A(BC) \\
&= (AB)C \\
&= I
\end{align*}

which shows that $A$ has an inverse $D$. Therefore, $A$ is invertible.

Similarly, define $E = CA$. Then:
\begin{align*}
EB &= (CA)B \\
&= C(AB) \\
&= I
\end{align*}

which shows that $B$ has an inverse $E$. Therefore, $B$ is invertible.

Thus, both $A$ and $B$ are invertible.

\item[(b)]

Let $A = \begin{pmatrix} 1 & 0 & 2 \\ 0 & 1 & 1 \end{pmatrix}$ and $B = \begin{pmatrix} 1 & 0 \\ 0 & 1 \\ 0 & 0 \end{pmatrix}$

Both $A$ and $B$ are not invertible (as they are not square matrices), but their product is:
\begin{align*}
AB &= \begin{pmatrix} 1 & 0 & 2 \\ 0 & 1 & 1 \end{pmatrix} \begin{pmatrix} 1 & 0 \\ 0 & 1 \\ 0 & 0 \end{pmatrix} \\
&= \begin{pmatrix} 1 & 0 \\ 0 & 1 \end{pmatrix}
\end{align*}

The product $AB$ is the $2 \times 2$ identity matrix, which is invertible.

\item[(c)]

Suppose, for contradiction, that $A$ is invertible. Then, multiplying both sides of $AB = 0$ by $A^{-1}$ on the left:
\begin{align*}
A^{-1}(AB) &= A^{-1}0 \\
(A^{-1}A)B &= 0 \\
IB &= 0 \\
B &= 0
\end{align*}

But this contradicts our assumption that $B$ is nonzero. Therefore, if $A$ and $B$ are nonzero $n \times n$ matrices with $AB = 0$, then $A$ cannot be invertible.
\end{enumerate}

\newpage

\section*{Exercise 6}
Find the inverse of the following matrix, and check your answer two different ways.
$$A = \begin{pmatrix} 2 & 1 & 2 \\ 1 & -1 & 0 \\ 4 & -2 & 1 \end{pmatrix}.$$

\textbf{Solution: } \\



\end{document}