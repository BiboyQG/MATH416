\documentclass{article}
\usepackage{graphicx}
\usepackage{amsmath}
\usepackage{array}
\usepackage{fancyhdr}
\usepackage{amssymb}
\usepackage[shortlabels]{enumitem}

\DeclareMathOperator{\R}{\mathbb R}

\pagestyle{fancy}
\fancyhead[L]{Banghao Chi}
\fancyhead[C]{Homework 3}
\fancyhead[R]{20th Feb}

\fancyfoot[C]{\thepage}

\renewcommand{\headrulewidth}{0.5pt}
\renewcommand{\footrulewidth}{0.5pt}

\begin{document}

\section*{Exercise 1}
Label the following statements as true or false.
\begin{itemize}
\item[(a)] The zero vector space has no basis.
\item[(b)] Every vector space that is generated by a finite set has a basis.
\item[(c)] Every vector space has a finite basis.
\item[(d)] A vector space cannot have more than one basis.
\item[(e)] If a vector space has a finite basis, then the number of vectors in every basis is the same.
\item[(f)] The dimension of $\mathcal{P}_n(\mathbb{R})$, polynomials of degree $\leq n$, is $n$.
\item[(g)] The dimension of $\mathcal{M}_{m\times n}(\mathbb{R})$ is $m + n$.
\item[(h)] Suppose that $V$ is a finite-dimensional vector space, that $S_1$ is a linearly independent subset of $V$, and that $S_2$ is a subset of $V$ that generates $V$. Then $S_1$ cannot contain more vectors than $S_2$.
\item[(i)] If $S$ generates the vector space $V$, then every vector in $V$ can be written as a linear combination of vectors in $S$ in only one way.
\item[(j)] Every subspace of a finite-dimensional space is finite-dimensional.
\item[(k)] If $V$ is a vector space having dimension $n$, then $V$ has exactly one subspace with dimension 0 and exactly one subspace with dimension $n$.
\item[(l)] If $V$ is a vector space having dimension $n$, and if $S$ is a subset of $V$ with $n$ vectors, then $S$ is linearly independent if and only if $S$ spans $V$.
\end{itemize}

\textbf{Solution:}
\begin{enumerate}[label=(\alph*)]
\item false
\item true
\item false
\item false
\item true
\item false
\item false
\item true
\item false
\item true
\item true
\item true
\end{enumerate}

\newpage

\section*{Exercise 2}
Determine which of the following sets are bases for $\mathbb{R}^3$.
\begin{itemize}
\item[(a)] $\{(1,0,-1),(2,5,1),(0,-4,3)\}$
\item[(b)] $\{(2,-4,1),(0,3,-1),(6,0,-1)\}$
\end{itemize}

\textbf{Solution:} \\

(a) For $\{(1,0,-1),(2,5,1),(0,-4,3)\}$
\begin{align*}
\begin{pmatrix}
1 & 2 & 0 \\
0 & 5 & -4 \\
-1 & 1 & 3
\end{pmatrix} &\xrightarrow{R_3 + R_1} 
\begin{pmatrix}
1 & 2 & 0 \\
0 & 5 & -4 \\
0 & 3 & 3
\end{pmatrix} \\
&\xrightarrow{R_3 - \frac{3}{5}R_2}
\begin{pmatrix}
1 & 2 & 0 \\
0 & 5 & -4 \\
0 & 0 & 5.4
\end{pmatrix}
\end{align*}

We get a row echelon form with no zero rows, therefore these vectors are linearly independent and form a basis for $\mathbb{R}^3$. \\

(b) For $\{(2,-4,1),(0,3,-1),(6,0,-1)\}$
\begin{align*}
\begin{pmatrix}
2 & 0 & 6 \\
-4 & 3 & 0 \\
1 & -1 & -1
\end{pmatrix} &\xrightarrow{R_2 + 2R_1}
\begin{pmatrix}
2 & 0 & 6 \\
0 & 3 & 12 \\
1 & -1 & -1
\end{pmatrix} \\
&\xrightarrow{R_3 - \frac{1}{2}R_1}
\begin{pmatrix}
2 & 0 & 6 \\
0 & 3 & 12 \\
0 & -1 & -4
\end{pmatrix} \\
&\xrightarrow{R_3 + \frac{1}{3}R_2}
\begin{pmatrix}
2 & 0 & 6 \\
0 & 3 & 12 \\
0 & 0 & 0
\end{pmatrix}
\end{align*}

We get a row echelon form with one zero row, therefore these vectors are linearly dependent and do not form a basis for $\mathbb{R}^3$.

\newpage

\section*{Exercise 3}
Let $W$ denote the subspace of $\mathbb{R}^5$ consisting of all the vectors having coordinates that sum to zero. The vectors
\begin{align*}
u_1 &= (2,-3,4,-5,2), \quad u_2 = (-6,9,-12,15,-6),\\
u_3 &= (3,-2,7,-9,1), \quad u_4 = (2,-8,2,-2,6),\\
u_5 &= (-1,1,2,1,-3), \quad u_6 = (0,-3,-18,9,12),\\
u_7 &= (1,0,-2,3,-2), \quad u_8 = (2,-1,1,-9,7)
\end{align*}
generate $W$. Find a subset of the set $\{u_1,u_2,\ldots,u_8\}$ that is a basis for $W$. \\

\textbf{Solution:} \\

Since $W$ is a subspace of $\mathbb{R}^5$ where the sum of coordinates is zero, 
one of the coordinates can be expressed as a linear combination of the others. Therefore:
\[\dim(W) = 5 - 1 = 4\]
meaning that the basis of $W$ has 4 vectors. \\

Note that $u_2 = -3u_1$, so $u_2$ is excluded. \\

Use the first four($u_1, u_3, u_4, u_5$) remaining vectors and check if they're linearly independent.
\begin{align*}
\begin{pmatrix}
2 & -3 & 4 & -5 & 2 \\
3 & -2 & 7 & -9 & 1 \\
2 & -8 & 2 & -2 & 6 \\
-1 & 1 & 2 & 1 & -3
\end{pmatrix}
&\xrightarrow{R_2 - \frac{3}{2}R_1}
\begin{pmatrix}
2 & -3 & 4 & -5 & 2 \\
0 & 2.5 & 1 & -1.5 & -2 \\
2 & -8 & 2 & -2 & 6 \\
-1 & 1 & 2 & 1 & -3
\end{pmatrix} \\
&\xrightarrow{R_3 - R_1}
\begin{pmatrix}
2 & -3 & 4 & -5 & 2 \\
0 & 2.5 & 1 & -1.5 & -2 \\
0 & -5 & -2 & 3 & 4 \\
-1 & 1 & 2 & 1 & -3
\end{pmatrix} \\
&\xrightarrow{R_4 + \frac{1}{2}R_1}
\begin{pmatrix}
2 & -3 & 4 & -5 & 2 \\
0 & 2.5 & 1 & -1.5 & -2 \\
0 & -5 & -2 & 3 & 4 \\
0 & -0.5 & 4 & -1.5 & -2
\end{pmatrix} \\
&\xrightarrow{R_3 + 2R_2}
\begin{pmatrix}
2 & -3 & 4 & -5 & 2 \\
0 & 2.5 & 1 & -1.5 & -2 \\
0 & 0 & 0 & 0 & 0 \\
0 & -0.5 & 4 & -1.5 & -2
\end{pmatrix} \\
&\xrightarrow{R_4 + 0.2R_2}
\begin{pmatrix}
2 & -3 & 4 & -5 & 2 \\
0 & 2.5 & 1 & -1.5 & -2 \\
0 & 0 & 0 & 0 & 0 \\
0 & 0 & 4.2 & -1.8 & -2.4
\end{pmatrix}
\end{align*}

The matrix has only 3 pivot columns, indicating that these vectors are not linearly independent. \\

Try $\{u_1, u_3, u_4, u_7\}$:
\begin{align*}
\begin{pmatrix}
2 & -3 & 4 & -5 & 2 \\
3 & -2 & 7 & -9 & 1 \\
-1 & 1 & 2 & 1 & -3 \\
1 & 0 & -2 & 3 & -2
\end{pmatrix}
&\xrightarrow{R_2 - \frac{3}{2}R_1}
\begin{pmatrix}
2 & -3 & 4 & -5 & 2 \\
0 & 2.5 & 1 & -1.5 & -2 \\
-1 & 1 & 2 & 1 & -3 \\
1 & 0 & -2 & 3 & -2
\end{pmatrix} \\
&\xrightarrow{R_3 + \frac{1}{2}R_1}
\begin{pmatrix}
2 & -3 & 4 & -5 & 2 \\
0 & 2.5 & 1 & -1.5 & -2 \\
0 & -0.5 & 4 & -1.5 & -2 \\
1 & 0 & -2 & 3 & -2
\end{pmatrix} \\
&\xrightarrow{R_4 - \frac{1}{2}R_1}
\begin{pmatrix}
2 & -3 & 4 & -5 & 2 \\
0 & 2.5 & 1 & -1.5 & -2 \\
0 & -0.5 & 4 & -1.5 & -2 \\
0 & 1.5 & -4 & 5.5 & -3
\end{pmatrix} \\
&\xrightarrow{R_3 + 0.2R_2}
\begin{pmatrix}
2 & -3 & 4 & -5 & 2 \\
0 & 2.5 & 1 & -1.5 & -2 \\
0 & 0 & 4.2 & -1.8 & -2.4 \\
0 & 1.5 & -4 & 5.5 & -3
\end{pmatrix} \\
&\xrightarrow{R_4 - 0.6R_2}
\begin{pmatrix}
2 & -3 & 4 & -5 & 2 \\
0 & 2.5 & 1 & -1.5 & -2 \\
0 & 0 & 4.2 & -1.8 & -2.4 \\
0 & 0 & -4.6 & 6.4 & -1.8
\end{pmatrix} \\
&\xrightarrow{R_4 + \frac{23}{21}R_3}
\begin{pmatrix}
2 & -3 & 4 & -5 & 2 \\
0 & 2.5 & 1 & -1.5 & -2 \\
0 & 0 & 4.2 & -1.8 & -2.4 \\
0 & 0 & 0 & 2.03 & -6.63
\end{pmatrix}
\end{align*}

The matrix has 4 pivot columns, indicating these vectors are linearly independent. Therefore, $\{u_1, u_3, u_5, u_7\}$ forms a basis for $W$.

\newpage

\section*{Exercise 4}
Recall from HW1.2 that, the subset $U$ of all upper triangular matrices in $\mathcal{M}_{n\times n}(\mathbb{R})$ forms a subspace. Find a basis for $U$ and use it to compute the dimension of $U$. \\

\textbf{Solution:} \\



\newpage

\section*{Exercise 5}
Suppose $W$ is a subspace of a finite-dimensional vector space $V$. For some $v \in V$ not in $W$, set $X = \text{span}(W \cup \{v\})$. Prove that $\dim(X) = \dim(W) + 1$. \\

\textbf{Solution:} \\

\newpage

\section*{Exercise 6}
Let $V$ be a vector space having dimension $n$, and let $S$ be a subset of $V$ that generates $V$.
\begin{itemize}
\item[(a)] Prove that there is a subset of $S$ that is a basis for $V$. (Be careful not to assume that $S$ is finite.)
\item[(b)] Prove that $S$ contains at least $n$ vectors.
\end{itemize}

\textbf{Solution:} \\



\end{document}