\documentclass{article}
\usepackage{graphicx}
\usepackage{amsmath}
\usepackage{array}
\usepackage{fancyhdr}
\usepackage{amssymb}
\usepackage[shortlabels]{enumitem}

\DeclareMathOperator{\R}{\mathbb R}

\pagestyle{fancy}
\fancyhead[L]{Banghao Chi}
\fancyhead[C]{Homework 7}
\fancyhead[R]{10th Apr}

\fancyfoot[C]{\thepage}

\renewcommand{\headrulewidth}{0.5pt}
\renewcommand{\footrulewidth}{0.5pt}

\begin{document}

\section*{Exercise 1}
Let $M$ be a square matrix of the form $\begin{pmatrix} A & B \\ 0 & I_n \end{pmatrix}$, for some matrix $A \in \mathcal{M}_{k\times k}(\mathbb{R})$, $B \in \mathcal{M}_{k\times n}$ and the identity matrix $I_n$. Using induction on $n$, prove that
\[\det(M) = \det(A).\]

\textbf{Solution:} \\

Base Case: $n = 1$

For $n = 1$, we have:
$$M_1 = \begin{pmatrix} A & B_1 \\ 0 & 1 \end{pmatrix}$$

By theorem of upper triangular matrices, we have:
$$\det(M_1) = \det(A) \cdot \det(1) = \det(A) \cdot 1 = \det(A)$$

Inductive Step:

Assume that for some $n = m$, we have:
$$\det(M_m) = \det\begin{pmatrix} A & B_m \\ 0 & I_m \end{pmatrix} = \det(A)$$

For $n = m+1$, split $B_{m+1}$ as $[B_m \; b_{m+1}]$. Similarly, split $I_{m+1}$ as:
$$I_{m+1} = \begin{pmatrix} I_m & 0 \\ 0 & 1 \end{pmatrix}$$

Therefore, we have:
$$M_{m+1} = \begin{pmatrix} A & B_m & b_{m+1} \\ 0 & I_m & 0 \\ 0 & 0 & 1 \end{pmatrix}$$

Using cofactor expansion along the last row:
\begin{align*}
\det(M_{m+1}) &= (-1)^{(k+m+1)+(k+m+1)} \cdot 1 \cdot \det(\tilde{M}_{m+1, m+1}) \\
&= \det(\tilde{M}_{m+1, m+1}) \\
&= \det\begin{pmatrix} A & B_m \\ 0 & I_m \end{pmatrix} \\
&= \det(A)
\end{align*}

Therefore, by the principle of mathematical induction, for any $n \geq 1$, the determinant of $M = \begin{pmatrix} A & B \\ 0 & I_n \end{pmatrix}$ equals $\det(A)$.

\newpage

\section*{Exercise 2}
For each of the following linear operators $T$ on a vector space $V$ and ordered bases $\beta$, compute $[T]_\beta$, and determine whether $\beta$ is a basis consisting of eigenvectors of $T$.

\begin{enumerate}
    \item[(a)] $V = \mathbb{R}^2$, $T(a,b) = (10a - 6b, 17a - 10b)$ and $\beta = \{(1,2),(2,3)\}$.
    \item[(b)] $V = \mathbb{R}^3$, $T(a,b,c) = (3a + 2b - 2c, -4a - 3b + 2c, -c)$ and $\beta = \{(0,1,1),(1,-1,0),(1,0,2)\}$.
\end{enumerate}

\textbf{Solution:} \\

(a) Let $v_1 = (1,2)$ and $v_2 = (2,3)$. \\

For $v_1 = (1,2)$:
\begin{align*}
    T(v_1) &= T(1,2) \\
    &= (10 \cdot 1 - 6 \cdot 2, 17 \cdot 1 - 10 \cdot 2) \\
    &= (10 - 12, 17 - 20) \\
    &= (-2, -3) \\
    &= -v_2
\end{align*}

For $v_2 = (2,3)$:
\begin{align*}
    T(v_2) &= T(2,3) \\
    &= (10 \cdot 2 - 6 \cdot 3, 17 \cdot 2 - 10 \cdot 3) \\
    &= (20 - 18, 34 - 30) \\
    &= (2, 4) \\
    &= 2v_1
\end{align*}

Therefore, the matrix $[T]_\beta$ is:
$[T]_\beta = \begin{pmatrix} 0 & 2 \\ -1 & 0 \end{pmatrix}$

For $v_1$: $T(v_1) = -v_2 \neq \lambda v_1$ for any $\lambda$ \\

For $v_2$: $T(v_2) = 2v_1 \neq \lambda v_2$ for any $\lambda$ \\

Therefore, $\beta$ is not a basis consisting of eigenvectors of $T$. \\

(b) Let $v_1 = (0,1,1)$, $v_2 = (1,-1,0)$, and $v_3 = (1,0,2)$. \\

For $v_1 = (0,1,1)$:
\begin{align*}
    T(v_1) &= T(0,1,1) \\
    &= (3 \cdot 0 + 2 \cdot 1 - 2 \cdot 1, -4 \cdot 0 - 3 \cdot 1 + 2 \cdot 1, -1 \cdot 1) \\
    &= (0, -1, -1) \\
    &= -v_1
\end{align*}

For $v_2 = (1,-1,0)$:
\begin{align*}
    T(v_2) &= T(1,-1,0) \\
    &= (3 \cdot 1 + 2 \cdot (-1) - 2 \cdot 0, -4 \cdot 1 - 3 \cdot (-1) + 2 \cdot 0, -1 \cdot 0) \\
    &= (1, -1, 0) \\
    &= v_2
\end{align*}

For $v_3 = (1,0,2)$:
\begin{align*}
    T(v_3) &= T(1,0,2) \\
    &= (3 \cdot 1 + 2 \cdot 0 - 2 \cdot 2, -4 \cdot 1 - 3 \cdot 0 + 2 \cdot 2, -1 \cdot 2) \\
    &= (-1, 0, -2) \\
    &= -v_3
\end{align*}

Therefore, the matrix $[T]_\beta$ is:
$[T]_\beta = \begin{pmatrix} -1 & 0 & 0 \\ 0 & 1 & 0 \\ 0 & 0 & -1 \end{pmatrix}$

For $v_1$: $T(v_1) = -v_1$, so $v_1$ is an eigenvector with eigenvalue $\lambda_1 = -1$ \\

For $v_2$: $T(v_2) = v_2$, so $v_2$ is an eigenvector with eigenvalue $\lambda_2 = 1$ \\

For $v_3$: $T(v_3) = -v_3$, so $v_3$ is an eigenvector with eigenvalue $\lambda_3 = -1$ \\

Therefore, $\beta$ is a basis consisting of eigenvectors of $T$.

\newpage

\section*{Exercise 3}
Let $A = \begin{pmatrix} 1 & 2 \\ 3 & 2 \end{pmatrix} \in \mathcal{M}_{2\times 2}(\mathbb{R})$.

\begin{enumerate}
    \item[(a)] Determine all the eigenvalues of $A$.

    \item[(b)] For each eigenvalue $\lambda$ of $A$, find the set of eigenvectors corresponding to $\lambda$.

    \item[(c)] If possible, find a basis for $\mathbb{R}^2$ consisting of eigenvectors of $A$.

    \item[(d)] If successful in finding such a basis, determine an invertible matrix $Q$ and a diagonal matrix $D$ such that $Q^{-1}AQ = D$.
\end{enumerate}

\textbf{Solution:} \\



\newpage

\section*{Exercise 4}
For each linear operator $T$ on $V$, find the eigenvalues of $T$ and an ordered basis $\beta$ for $V$ such that $[T]_\beta$ is a diagonal matrix.

\begin{enumerate}
    \item[(a)] $V = \mathbb{R}^3$ and $T(a,b,c) = (7a - 4b + 10c, 4a - 3b + 8c, -2a + b - 2c)$.

    \item[(b)] $V = \mathcal{M}_{2\times 2}(\mathbb{R})$ and $T\begin{pmatrix} a & b \\ c & d \end{pmatrix} = \begin{pmatrix} d & b \\ c & a \end{pmatrix}$.
\end{enumerate}

\textbf{Solution:} \\



\newpage

\section*{Exercise 5}
Let $T$ be a linear operator on a finite-dimensional vector space $V$.

\begin{enumerate}
    \item[(a)] Show that $T$ is invertible if and only if $0$ is not an eigenvalue of $T$.

    \item[(b)] If $T$ is invertible, show that $\lambda^{-1}$ is an eigenvalue of $T^{-1}$ if and only if $\lambda$ is an eigenvalue of $T$.
\end{enumerate}

\textbf{Solution:} \\



\newpage

\section*{Exercise 6}
Suppose $T : V \rightarrow V$ is a linear operator on a finite-dimensional vector space $V$. Suppose $v \in V$ is an eigenvector of $T$ with eigenvalue $\lambda$. As usual, $T^m : V \rightarrow V$ denotes composition of $T$ with itself $m$ times. Prove that $v$ is also an eigenvector for $T^m$ and give a formula for the corresponding eigenvalue. \\

\textbf{Solution:} \\



\newpage

\end{document}