\documentclass{article}
\usepackage{graphicx}
\usepackage{amsmath}
\usepackage{array}
\usepackage{fancyhdr}
\usepackage{amssymb}
\usepackage[shortlabels]{enumitem}

\DeclareMathOperator{\R}{\mathbb R}

\pagestyle{fancy}
\fancyhead[L]{Banghao Chi}
\fancyhead[C]{Homework 1}
\fancyhead[R]{25th Jan}

\fancyfoot[C]{\thepage}

\renewcommand{\headrulewidth}{0.5pt}
\renewcommand{\footrulewidth}{0.5pt}

\begin{document}

\section*{Exercise 1}
Determine whether the following sets are subspaces of $\mathbb{R}^3$ under the operations of addition and scalar multiplication defined on $\mathbb{R}^3$. Justify your answers.
\begin{enumerate}[label=(\alph*)]
\item $W_1 = \{(a_1,a_2,a_3) \in \mathbb{R}^3 : a_1 = 3a_2, a_3 = -a_2\}$
\item $W_2 = \{(a_1,a_2,a_3) \in \mathbb{R}^3 : a_1 = a_3 + 2\}$
\item $W_3 = \{(a_1,a_2,a_3) \in \mathbb{R}^3 : 2a_1 - 7a_2 + a_3 = 0\}$
\item $W_4 = \{(a_1,a_2,a_3) \in \mathbb{R}^3 : a_1 - 4a_2 - a_3 = 0\}$
\item $W_5 = \{(a_1,a_2,a_3) \in \mathbb{R}^3 : a_1 + 2a_2 - 3a_3 = 1\}$
\end{enumerate}
Describe $W_1 \cap W_3, W_1 \cap W_4$ and $W_3 \cap W_4$, and observe that each is a subspace of $\mathbb{R}^3$. \\

\textbf{Solution:}

By definition of a subspace, a non-empty subset $W \subseteq \mathbb{R}^n$ is a subspace if and only if:
\begin{enumerate}
    \item The zero vector $\mathbf{0}$ is in $W$.
    \item $W$ is closed under addition: if $\mathbf{u}, \mathbf{v} \in W$, then $\mathbf{u} + \mathbf{v} \in W$.
    \item $W$ is closed under scalar multiplication: if $\mathbf{u} \in W$ and $\alpha \in \mathbb{R}$, then $\alpha \mathbf{u} \in W$.
\end{enumerate}

\bigskip

\noindent
\textbf{(a)} $W_1 = \{(a_1,a_2,a_3) \in \mathbb{R}^3 : a_1 = 3a_2,\, a_3 = -a_2\}$. \\
\noindent
Any vector in $W_1$ can be written as
\[
(a_1, a_2, a_3) = (3a_2,\, a_2,\, -a_2).
\]
Let $t = a_2$. Then every vector in $W_1$ has the form
\[
(3t,\, t,\, -t) = t\,(3, 1, -1).
\]
meaning we can write $W_1$ as
\[
W_1 = \{t(3,1,-1) : t \in \mathbb{R}\}.
\]
Now we check the three conditions for $W_1$ to be a subspace:
\begin{itemize}
    \item The zero vector $(0,0,0)$ is in $W_1$, because when $t = 0$, we have $0(3,1,-1) = (0,0,0)$.
    \item $W_1$ is closed under addition because for any two vectors $\mathbf{v}_1 = (3a_2,a_2,-a_2) \in W_1$ and $\mathbf{v}_2 = (3b_2,b_2,-b_2) \in W_1$, we have
\begin{align*}
\mathbf{v}_1 + \mathbf{v}_2 &= (3a_2,a_2,-a_2) + (3b_2,b_2,-b_2) \\
                              &= (3(a_2+b_2),(a_2+b_2),-(a_2+b_2)) \\
                              &= (a_2 + b_2)(3,1,-1) \in W_1.
\end{align*}
    \item $W_1$ is closed under scalar multiplication because for any vector $\mathbf{v} = (3a_2,a_2,-a_2) \in W_1$ and any scalar $c \in \mathbb{R}$, we have
\begin{align*}
c\mathbf{v} &= c(3a_2,a_2,-a_2) \\
            &= (3ca_2,ca_2,-ca_2) \\
            &= t(3c,c,-c) \\
            &= tc(3,1,-1) \in W_1.
\end{align*}
\end{itemize}
Thus $W_1$ satisfies all three conditions for being a subspace of $\mathbb{R}^3$. Therefore $W_1 \text{ is a subspace of } \mathbb{R}^3$.

\bigskip

\noindent
\textbf{(b)} $W_2 = \{(a_1,a_2,a_3) \in \mathbb{R}^3 : a_1 = a_3 + 2\}$.\\
\noindent
Check if the zero vector $(0,0,0)$ is in $W_2$. If $(0,0,0)$ were in $W_2$, we would need
\[
a_1 = 0, \quad a_3 = 0 \quad \Longrightarrow \quad 0 = 0 + 2,
\]
which is false since that implies $0=2$. Therefore $(0,0,0) \notin W_2$. Since a subspace must contain the zero vector, $W_2$ is not a subspace of $\mathbb{R}^3$.

\bigskip

\noindent
\textbf{(c)} $W_3 = \{(a_1,a_2,a_3) \in \mathbb{R}^3 : 2a_1 - 7a_2 + a_3 = 0\}$.

\begin{itemize}
\item Check that $\mathbf{0} = (0,0,0)$ can be satisfied when $a_1 = a_2 = a_3 = 0$:
\[
2\cdot 0 - 7\cdot 0 + 0 = 0.
\]
Hence $\mathbf{0} \in W_3$.

\item Check if $W_3$ is closed under addition. Let $\mathbf{u} = (a_1,a_2,a_3) \in W_3$ and $\mathbf{v} = (b_1,b_2,b_3) \in W_3$. Then
\[
\left\{\begin{aligned}
2a_1 - 7a_2 + a_3 &= 0, \\
2b_1 - 7b_2 + b_3 &= 0.
\end{aligned}\right.
\]
Adding these two equations, we get
\[
2(a_1 + b_1) - 7(a_2 + b_2) + (a_3 + b_3) = 0.
\]
Now, adding $\mathbf{u}$ and $\mathbf{v}$, we get
\begin{align*}
\mathbf{u} + \mathbf{v} &= (a_1, a_2, a_3) + (b_1, b_2, b_3) \\
&= (a_1 + b_1, a_2 + b_2, a_3 + b_3)
\end{align*}
Since $2(a_1 + b_1) - 7(a_2 + b_2) + (a_3 + b_3) = 0$, we have $\mathbf{u} + \mathbf{v} \in W_3$. Hence $W_3$ is closed under addition.

\item Check if $W_3$ is closed under scalar multiplication. Let $\mathbf{u} = (a_1,a_2,a_3) \in W_3$ and $\alpha \in \mathbb{R}$. Then
\[
2a_1 - 7a_2 + a_3 = 0.
\]
Multiplying this equation by $\alpha$, we get
\[
2\alpha a_1 - 7\alpha a_2 + \alpha a_3 = 0.
\]
Now, multiplying $\mathbf{u}$ by $\alpha$, we get
\begin{align*}
\alpha \mathbf{u} &= \alpha (a_1, a_2, a_3) \\
&= (\alpha a_1, \alpha a_2, \alpha a_3)
\end{align*}
Since $2\alpha a_1 - 7\alpha a_2 + \alpha a_3 = 0$, we have $\alpha \mathbf{u} \in W_3$. Hence $W_3$ is closed under scalar multiplication.
\end{itemize}

Therefore $W_3 \text{ is a subspace of } \mathbb{R}^3$.

\bigskip

\noindent
\textbf{(d)} $W_4 = \{(a_1,a_2,a_3) \in \mathbb{R}^3 : a_1 - 4a_2 - a_3 = 0\}$.

\begin{itemize}
\item Check that $\mathbf{0} = (0,0,0)$ can be satisfied when $a_1 = a_2 = a_3 = 0$:
\[
a_1 - 4a_2 - a_3 = 0.
\]
Hence $\mathbf{0} \in W_4$.

\item Check if $W_4$ is closed under addition. Let $\mathbf{u} = (a_1,a_2,a_3) \in W_4$ and $\mathbf{v} = (b_1,b_2,b_3) \in W_4$. Then
\[
\left\{\begin{aligned}
a_1 - 4a_2 - a_3 &= 0, \\
b_1 - 4b_2 - b_3 &= 0.
\end{aligned}\right.
\]
Adding these two equations, we get
\[
a_1 + b_1 - 4(a_2 + b_2) - (a_3 + b_3) = 0.
\]
Now, adding $\mathbf{u}$ and $\mathbf{v}$, we get
\begin{align*}
\mathbf{u} + \mathbf{v} &= (a_1, a_2, a_3) + (b_1, b_2, b_3) \\
&= (a_1 + b_1, a_2 + b_2, a_3 + b_3)
\end{align*}
Since $a_1 + b_1 - 4(a_2 + b_2) - (a_3 + b_3) = 0$, we have $\mathbf{u} + \mathbf{v} \in W_4$. Hence $W_4$ is closed under addition.

\item Check if $W_4$ is closed under scalar multiplication. Let $\mathbf{u} = (a_1,a_2,a_3) \in W_4$ and $\alpha \in \mathbb{R}$. Then
\[
a_1 - 4a_2 - a_3 = 0.
\]
Multiplying this equation by $\alpha$, we get
\[
\alpha a_1 - 4\alpha a_2 - \alpha a_3 = 0.
\]
Now, multiplying $\mathbf{u}$ by $\alpha$, we get
\begin{align*}
\alpha \mathbf{u} &= \alpha (a_1, a_2, a_3) \\
&= (\alpha a_1, \alpha a_2, \alpha a_3)
\end{align*}
Since $\alpha a_1 - 4\alpha a_2 - \alpha a_3 = 0$, we have $\alpha \mathbf{u} \in W_4$. Hence $W_4$ is closed under scalar multiplication.
\end{itemize}

Therefore $W_4 \text{ is a subspace of } \mathbb{R}^3$.

\bigskip

\noindent
\textbf{(e)} $W_5 = \{(a_1,a_2,a_3) \in \mathbb{R}^3 : a_1 + 2a_2 - 3a_3 = 1\}$.

Check the zero vector:
\[
0 + 2\cdot0 - 3\cdot0 = 0 \neq 1,
\]
so $\mathbf{0}\notin W_5$. Hence $W_5$ is not a subspace. Therefore, $W_5 \text{ is not a subspace of } \mathbb{R}^3$.

\bigskip

\noindent
\textbf{Conclusion on subspaces:}
\[
W_1, W_3, \text{ and } W_4 \text{ are subspaces}; \quad W_2 \text{ and } W_5 \text{ are not subspaces.}
\]

\bigskip

\noindent
\textbf{Intersections:} We next describe $W_1\cap W_3$, $W_1\cap W_4$, and $W_3\cap W_4$. Note that the intersection of subspaces is always a subspace as well.

\medskip
\noindent
$W_1 \cap W_3$:
\[
W_1 = \{(3t,\, t,\, -t)\,:\, t\in \mathbb{R}\}, 
\quad
W_3 = \{(a_1,a_2,a_3)\,:\, 2a_1 -7a_2 + a_3 = 0\}.
\]
A vector in $W_1 \cap W_3$ must be of the form $(3t, t, -t)$ and also satisfy
\[
2(3t) - 7t + (-t) = 0
\quad \Longrightarrow \quad 6t - 7t - t = -2t = 0
\quad \Longrightarrow \quad t=0.
\]
Hence the only vector in $W_1 \cap W_3$ is $(0,0,0)$. Therefore $W_1 \cap W_3 = \{\mathbf{0}\}$.

\medskip
\noindent
$W_1 \cap W_4$:
\[
W_1 = \{(3t,\, t,\, -t)\,:\, t\in \mathbb{R}\}, 
\quad
W_4 = \{(a_1,a_2,a_3)\,:\, a_1 -4a_2 - a_3 = 0\}.
\]
We require $(3t, t, -t)$ to satisfy
\[
(3t) - 4t - (-t) = 3t - 4t + t = 0.
\]
This simplifies to $0=0$ for all $t$. Hence \emph{every} vector in $W_1$ also lies in $W_4$. Therefore $W_1 \cap W_4 = W_1 = \{(3t, t, -t) : t \in \mathbb{R}\}$.

\medskip
\noindent
$W_3 \cap W_4$:
\[
W_3 = \{(a_1,a_2,a_3) : 2a_1 - 7a_2 + a_3 = 0\}, 
\quad
W_4 = \{(a_1,a_2,a_3) : a_1 - 4a_2 - a_3 = 0\}.
\]
\[
\begin{cases}
2a_1 - 7a_2 + a_3 = 0,\\
a_1 - 4a_2 - a_3 = 0.
\end{cases}
\]
We can solve this system as follows. From the second equation,
\[
a_3 = a_1 - 4a_2.
\]
Substitute $a_3$ into the first:
\[
2a_1 - 7a_2 + (a_1 - 4a_2) = 0 
\quad \Longrightarrow \quad 3a_1 - 11a_2 = 0 
\quad \Longrightarrow \quad 3a_1 = 11a_2.
\]
Thus $a_1 = \tfrac{11}{3}\,a_2$. Then
\[
a_3 = a_1 - 4a_2 = \frac{11}{3}a_2 \;-\; 4a_2 
= \frac{11}{3}a_2 - \frac{12}{3}a_2 
= -\frac{1}{3}a_2.
\]
Hence any vector in $W_3 \cap W_4$ has the form
\[
\left(\tfrac{11}{3}a_2,\; a_2,\; -\tfrac{1}{3}a_2\right).
\]
Set $a_2 = 3s$ for convenience, to clear denominators. Then
\[
a_1 = \frac{11}{3}(3s) = 11s,\quad
a_2 = 3s,\quad
a_3 = -\frac{1}{3}(3s) = -s.
\]
Thus
\[
(a_1, a_2, a_3) = (11s,\,3s,\,-s) = s\,(11,\,3,\,-1).
\]
Hence $W_3 \cap W_4 = \{ s(11,3,-1) : s \in \mathbb{R}\}$.

\bigskip

\noindent
\textbf{Final summary:}
\[
\begin{aligned}
&\bullet\; W_1,\,W_3,\,W_4 \text{ are subspaces.}
\quad\bullet\; W_2,\,W_5 \text{ are not subspaces.} \\[6pt]
&\bullet\; W_1 \cap W_3 = \{\mathbf{0}\}, 
\quad 
W_1 \cap W_4 = W_1, 
\quad 
W_3 \cap W_4 = \text{span}\{s(11,\,3,\, -1): s \in \mathbb{R}\}.
\end{aligned}
\]

\newpage

\section*{Exercise 2}
A square matrix $A$ is called upper triangular if all entries lying below the diagonal are 0, that is, $A_{ij} = 0$ whenever $i > j$. Show that the upper triangular matrices form a subspace of $\mathcal{M}_{n\times n}(\mathbb{R})$. \\

\textbf{Solution:}

To show that the upper triangular matrices form a subspace of $\mathcal{M}_{n\times n}(\mathbb{R})$, by definition, we need to show that they include zero matrix, are closed under addition, and are closed under scalar multiplication.

\begin{itemize}
\item The zero matrix is an upper triangular matrix. Check:
\[
\begin{pmatrix}
0 & \cdots & 0 \\
\vdots & \ddots & \vdots \\
0 & \cdots & 0
\end{pmatrix}
\]
\item If $A$ and $B$ are upper triangular matrices, then $A + B$ is also an upper triangular matrix. Check:
Let $A$ and $B$ be upper triangular matrices:
\[
A = \begin{pmatrix}
a_{11} & \cdots & a_{1n} \\
\vdots & \ddots & \vdots \\
0 & \cdots & a_{nn}
\end{pmatrix}
\]
\[
B = \begin{pmatrix}
b_{11} & \cdots & b_{1n} \\
\vdots & \ddots & \vdots \\
0 & \cdots & b_{nn}
\end{pmatrix}
\]
Then $A + B$ is also an upper triangular matrix of the form:
\[
\begin{pmatrix}
a_{11} + b_{11} & \cdots & a_{1n} + b_{1n} \\
\vdots & \ddots & \vdots \\
0 & \cdots & a_{nn} + b_{nn}
\end{pmatrix}
\]
\item If $A$ is an upper triangular matrix and $c$ is a scalar, then $cA$ is also an upper triangular matrix. Check:
Let $A$ be an upper triangular matrix:
\[
A = \begin{pmatrix}
a_{11} & \cdots & a_{1n} \\
\vdots & \ddots & \vdots \\
0 & \cdots & a_{nn}
\end{pmatrix}
\]
Then $cA$ is also an upper triangular matrix of the form:
\[
\begin{pmatrix}
ca_{11} & \cdots & ca_{1n} \\
\vdots & \ddots & \vdots \\
0 & \cdots & ca_{nn}
\end{pmatrix}
\]
\end{itemize}

Therefore, the upper triangular matrices form a subspace of $\mathcal{M}_{n\times n}(\mathbb{R})$.


\newpage

\section*{Exercise 3}
Solve the following systems of linear equations by adding and multiplying equations.
\begin{enumerate}[label=(\alph*)]
\item $\begin{aligned}
2x_1&-2x_2&-3x_3& &= &-2 \\
3x_1&-3x_2&-2x_3&+5x_4 &= &7 \\
x_1&-x_2&-2x_3&-x_4 &= &-3
\end{aligned}$

\item $\begin{aligned}
3x_1&-7x_2&+4x_3 &= 10 \\
x_1&-2x_2&+x_3 &= 3 \\
2x_1&-x_2&-2x_3 &= 6
\end{aligned}$
\end{enumerate}

\textbf{Solution:} \\

\begin{enumerate}[label=(\alph*)]
\item 
\[\left\{
\begin{aligned}
2x_1&-2x_2&-3x_3& &= &-2 &\quad (e_1)\\
3x_1&-3x_2&-2x_3&+5x_4 &= &7 &\quad (e_2)\\
x_1&-x_2&-2x_3&-x_4 &= &-3 &\quad (e_3)
\end{aligned}
\right.
\]

By observation, we can see that $e_2 = 4e_1 - 5e_3$. This means that we only need to solve the system of equations consisted of $e_1$ and $e_3$.

\[\left\{
\begin{aligned}
2x_1&-2x_2&-3x_3& &= &-2 &\quad (e_1)\\
x_1&-x_2&-2x_3&-x_4 &= &-3 &\quad (e_3)
\end{aligned}
\right.
\]

Let $e_1 = e_1 - 2e_3$, we get:
\[\left\{
\begin{aligned}
&&x_3&+2x_4&= & 4&\quad (e_1)\\
x_1&-x_2&-2x_3&-x_4 &= &-3 &\quad (e_3)
\end{aligned}
\right.
\]

Let $e_3 = e_3 + 2e_4$, we get:
\[\left\{
\begin{aligned}
&&x_3&+2x_4&= & 4&\quad (e_1)\\
x_1&-x_2& &+3x_4&= & 5&\quad (e_3)
\end{aligned}
\right.
\]

Set $x_2 = a$, $x_4 = b$, we get:
\[\left\{
\begin{aligned}
&&x_3&+2b&= & 4&\quad (e_1)\\
x_1&-a& &+3b &= &5 &\quad (e_3)
\end{aligned}
\right.
\]

Simplify the system, we get:
\[\left\{
\begin{aligned}
x_3 &= 4 - 2b\\
x_1 &= 5 - 3b + a
\end{aligned}
\right.
\]

Hence, the solution to the system is:
\[
(x_1, x_2, x_3, x_4) = (5 - 3b + a, a, 4 - 2b, b), \quad a, b \in \mathbb{R}
\]

\item 
\[
\left\{
\begin{aligned}
3x_1&-7x_2&+4x_3 &= 10 &\quad (e_1)\\
x_1&-2x_2&+x_3 &= 3 &\quad (e_2)\\
2x_1&-x_2&-2x_3 &= 6 &\quad (e_3)
\end{aligned}
\right.
\]
\end{enumerate}

Switching $e_1$ and $e_2$, we get:
\[
\left\{
\begin{aligned}
x_1&-2x_2&+x_3 &= 3 &\quad (e_2)\\
3x_1&-7x_2&+4x_3 &= 10 &\quad (e_1)\\
2x_1&-x_2&-2x_3 &= 6 &\quad (e_3)
\end{aligned}
\right.
\]

Let $e_1 = -(e_1 - 3e_2)$, $e_3 = e_3 - 2e_2$, we get:
\[
\left\{
\begin{aligned}
x_1&-2x_2&+x_3 &= 3 &\quad (e_2)\\
&x_2&-x_3&= -1 &\quad (e_1)\\
&3x_2&-4x_3&= 0 &\quad (e_3)
\end{aligned}
\right.
\]

Let $e_2 = e_2 + 2e_1$, $e_3 = e_3 - 3e_1$, we get:
\[
\left\{
\begin{aligned}
x_1&&-x_3 &= 1 &\quad (e_2)\\
&x_2&-x_3&= -1 &\quad (e_1)\\
&&-x_3&= 3 &\quad (e_3)
\end{aligned}
\right.
\]

We can see that $x_3 = -3$, plug it into $e_1$, we first get $x_2 = -4$, then plug both of them into $e_2$, we get $x_1 = -2$.

Hence, the solution to the system is:
\[
(x_1, x_2, x_3) = (-2, -4, -3)
\]

\newpage

\section*{Exercise 4}
For a nonempty set $S$, we use $\mathcal{F}(S,\mathbb{R})$ to denote the set of all functions from $S$ to $\mathbb{R}$; in class we have discussed the case $S = [-1,1]$. In fact, for any arbitrary set $S$, $\mathcal{F}(S,\mathbb{R})$ is a vector space over $\mathbb{R}$. Fix a point $s_0$ in $S$ and consider the subset $W$ of $\mathcal{F}(S,\mathbb{R})$ consisting of all functions where $f(s_0) = 0$.
\begin{enumerate}[label=(\alph*)]
\item Show that $W$ is a subspace of $\mathcal{F}(S,\mathbb{R})$.
\item Consider instead the subset where $f(s_0) = 1$. Is this also a subspace? Justify your answer.
\end{enumerate}

\textbf{Solution:}
\newpage

\section*{Exercise 5}
In these questions, determine whether the first vector can be expressed as a linear combination of the other two and write the coefficients or the associated scalars if the answer is yes.
\begin{enumerate}[label=(\alph*)]
\item $(-2,0,3),(1,3,0),(2,4,-1)$
\item $x^3 - 3x + 5, x^3 + 2x^2 - x + 1, x^3 + 3x^2 - 1$
\end{enumerate}

\textbf{Solution:}
\newpage

\section*{Exercise 6}
Suppose that $A$, $B$ and $C$, are $m \times n$ matrices with real coefficients. Prove the following three facts from the definition of row equivalence.
\begin{enumerate}[label=(\alph*)]
\item $A$ is row equivalent to $A$.
\item If $A$ is row equivalent to $B$, then $B$ is row equivalent to $A$.
\item If $A$ is row equivalent to $B$, and $B$ is row equivalent to $C$, then $A$ is row equivalent to $C$.
\end{enumerate}
Note: A relationship that satisfies these three properties is known as an equivalence relation; this is a formal way of saying that a relationship behaves like equality, without requiring the relationship to be as strict as equality itself. \\

\textbf{Solution:}

\end{document}