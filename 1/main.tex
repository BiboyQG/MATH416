\documentclass{article}
\usepackage{graphicx}
\usepackage{amsmath}
\usepackage{array}
\usepackage{fancyhdr}
\usepackage{amssymb}
\usepackage[shortlabels]{enumitem}

\DeclareMathOperator{\R}{\mathbb R}

\pagestyle{fancy}
\fancyhead[L]{Banghao Chi}
\fancyhead[C]{Homework 1}
\fancyhead[R]{25th Jan}

\fancyfoot[C]{\thepage}

\renewcommand{\headrulewidth}{0.5pt}
\renewcommand{\footrulewidth}{0.5pt}

\begin{document}

\section*{Exercise 1}
Determine whether the following sets are subspaces of $\mathbb{R}^3$ under the operations of addition and scalar multiplication defined on $\mathbb{R}^3$. Justify your answers.
\begin{enumerate}[label=(\alph*)]
\item $W_1 = \{(a_1,a_2,a_3) \in \mathbb{R}^3 : a_1 = 3a_2, a_3 = -a_2\}$
\item $W_2 = \{(a_1,a_2,a_3) \in \mathbb{R}^3 : a_1 = a_3 + 2\}$
\item $W_3 = \{(a_1,a_2,a_3) \in \mathbb{R}^3 : 2a_1 - 7a_2 + a_3 = 0\}$
\item $W_4 = \{(a_1,a_2,a_3) \in \mathbb{R}^3 : a_1 - 4a_2 - a_3 = 0\}$
\item $W_5 = \{(a_1,a_2,a_3) \in \mathbb{R}^3 : a_1 + 2a_2 - 3a_3 = 1\}$
\end{enumerate}
Describe $W_1 \cap W_3, W_1 \cap W_4$ and $W_3 \cap W_4$, and observe that each is a subspace of $\mathbb{R}^3$. \\

\textbf{Solution:}
\newpage

\section*{Exercise 2}
A square matrix $A$ is called upper triangular if all entries lying below the diagonal are 0, that is, $A_{ij} = 0$ whenever $i > j$. Show that the upper triangular matrices form a subspace of $\mathcal{M}_{n\times n}(\mathbb{R})$. \\

\textbf{Solution:}
\newpage

\section*{Exercise 3}
Solve the following systems of linear equations by adding and multiplying equations.
\begin{enumerate}[label=(\alph*)]
\item $\begin{aligned}
2x_1&-2x_2&-3x_3& &= &-2 \\
3x_1&-3x_2&-2x_3&+5x_4 &= &7 \\
x_1&-x_2&-2x_3&-x_4 &= &-3
\end{aligned}$

\item $\begin{aligned}
3x_1&-7x_2&+4x_3 &= 10 \\
x_1&-2x_2&+x_3 &= 3 \\
2x_1&-x_2&-2x_3 &= 6
\end{aligned}$
\end{enumerate}

\textbf{Solution:}
\newpage

\section*{Exercise 4}
For a nonempty set $S$, we use $\mathcal{F}(S,\mathbb{R})$ to denote the set of all functions from $S$ to $\mathbb{R}$; in class we have discussed the case $S = [-1,1]$. In fact, for any arbitrary set $S$, $\mathcal{F}(S,\mathbb{R})$ is a vector space over $\mathbb{R}$. Fix a point $s_0$ in $S$ and consider the subset $W$ of $\mathcal{F}(S,\mathbb{R})$ consisting of all functions where $f(s_0) = 0$.
\begin{enumerate}[label=(\alph*)]
\item Show that $W$ is a subspace of $\mathcal{F}(S,\mathbb{R})$.
\item Consider instead the subset where $f(s_0) = 1$. Is this also a subspace? Justify your answer.
\end{enumerate}

\textbf{Solution:}
\newpage

\section*{Exercise 5}
In these questions, determine whether the first vector can be expressed as a linear combination of the other two and write the coefficients or the associated scalars if the answer is yes.
\begin{enumerate}[label=(\alph*)]
\item $(-2,0,3),(1,3,0),(2,4,-1)$
\item $x^3 - 3x + 5, x^3 + 2x^2 - x + 1, x^3 + 3x^2 - 1$
\end{enumerate}

\textbf{Solution:}
\newpage

\section*{Exercise 6}
Suppose that $A$, $B$ and $C$, are $m \times n$ matrices with real coefficients. Prove the following three facts from the definition of row equivalence.
\begin{enumerate}[label=(\alph*)]
\item $A$ is row equivalent to $A$.
\item If $A$ is row equivalent to $B$, then $B$ is row equivalent to $A$.
\item If $A$ is row equivalent to $B$, and $B$ is row equivalent to $C$, then $A$ is row equivalent to $C$.
\end{enumerate}
Note: A relationship that satisfies these three properties is known as an equivalence relation; this is a formal way of saying that a relationship behaves like equality, without requiring the relationship to be as strict as equality itself. \\

\textbf{Solution:}

\end{document}