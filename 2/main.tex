\documentclass{article}
\usepackage{graphicx}
\usepackage{amsmath}
\usepackage{array}
\usepackage{fancyhdr}
\usepackage{amssymb}
\usepackage[shortlabels]{enumitem}

\DeclareMathOperator{\R}{\mathbb R}

\pagestyle{fancy}
\fancyhead[L]{Banghao Chi}
\fancyhead[C]{Homework 2}
\fancyhead[R]{13th Feb}

\fancyfoot[C]{\thepage}

\renewcommand{\headrulewidth}{0.5pt}
\renewcommand{\footrulewidth}{0.5pt}

\begin{document}

\section*{Exercise 1}
Solve each of the following linear systems by writing down its augmented matrix, doing row operations to get a matrix in reduced row echelon form, and using that to find all of the solutions. You should label your row operations as in section RREF of [Beezer]. \\

(a) 
\begin{align*}
2x_1 + x_2 &= 0 \\
x_1 + x_2 &= 1 \\
3x_1 + 4x_2 &= 5 \\
3x_1 + 5x_2 &= 7
\end{align*}

(b)
\begin{align*}
y_1 + 2y_2 - y_3 &= 1 \\
y_1 + y_2 + 2y_3 &= 0 \\
5y_1 + 8y_2 + y_3 &= 1
\end{align*}

(c)
\begin{align*}
2x_1 + 4x_2 + 5x_3 + 7x_4 &= 18 \\
x_1 + 2x_2 + x_3 - x_4 &= 3 \\
4x_1 + 8x_2 + 7x_3 + 5x_4 &= 24
\end{align*}

\textbf{Solution:} \\

(a) Augmented matrix:
\[
\left[
\begin{array}{cc|c}
2 & 1 & 0 \\
1 & 1 & 1 \\
3 & 4 & 5 \\
3 & 5 & 7
\end{array}
\right].
\]

Perform row operations to get the matrix in reduced row echelon form:

\begin{align*}
& R_1 \leftrightarrow R_2 
\quad \longrightarrow 
\quad
\left[
\begin{array}{cc|c}
1 & 1 & 1 \\
2 & 1 & 0 \\
3 & 4 & 5 \\
3 & 5 & 7
\end{array}
\right] \\
& R_2 \leftarrow R_2 - 2R_1 
\quad \longrightarrow 
\quad
\left[
\begin{array}{cc|c}
1 & 1 & 1\\
0 & -1 & -2\\
3 & 4 & 5\\
3 & 5 & 7
\end{array}
\right] \\
& R_3 \leftarrow R_3 - 3R_1 
\quad \longrightarrow 
\quad
\left[
\begin{array}{cc|c}
1 & 1 & 1 \\
0 & -1 & -2 \\
0 & 1 & 2 \\
3 & 5 & 7
\end{array}
\right] \\
& R_4 \leftarrow R_4 - 3R_1 
\quad \longrightarrow 
\quad
\left[
\begin{array}{cc|c}
1 & 1 & 1 \\
0 & -1 & -2 \\
0 & 1 & 2 \\
0 & 2 & 4
\end{array}
\right] \\
& R_2 \leftarrow -\,R_2 
\quad \longrightarrow 
\quad
\left[
\begin{array}{cc|c}
1 & 1 & 1 \\
0 & 1 & 2 \\
0 & 1 & 2 \\
0 & 2 & 4
\end{array}
\right] \\
& R_1 \leftarrow R_1 - R_2 
\quad R_3 \leftarrow R_3 - R_2 
\quad R_4 \leftarrow R_4 - 2R_2 
\\
&\qquad \longrightarrow 
\quad
\left[
\begin{array}{cc|c}
1 & 0 & -1 \\
0 & 1 & 2 \\
0 & 0 & 0 \\
0 & 0 & 0
\end{array}
\right].
\end{align*}

Hence, we have the system:
\[
\begin{cases}
x_1 = -1 \\
x_2 = 2
\end{cases}
\]

Thus the unique solution is
\[
(x_1, x_2) = (-1,\, 2).
\]

\bigskip

(b) Augmented matrix:
\[
\left[
\begin{array}{ccc|c}
1 & 2 & -1 & 1 \\
1 & 1 & 2 & 0 \\
5 & 8 & 1 & 1
\end{array}
\right]
\]

Perform row operations to get the matrix in reduced row echelon form:

\begin{align*}
& R_2 \leftarrow R_2 - R_1 
\quad \longrightarrow 
\quad
\left[
\begin{array}{ccc|c}
1 & 2 & -1 & 1 \\
0 & -1 & 3 & -1 \\
5 & 8 & 1 & 1
\end{array}
\right] \\
& R_3 \leftarrow R_3 - 5R_1 
\quad \longrightarrow 
\quad
\left[
\begin{array}{ccc|c}
1 & 2 & -1 & 1 \\
0 & -1 & 3 & -1 \\
0 & -2 & 6 & -4
\end{array}
\right] \\
& R_2 \leftarrow -\,R_2 
\quad \longrightarrow 
\quad
\left[
\begin{array}{ccc|c}
1 & 2 & -1 & 1 \\
0 & 1 & -3 & 1 \\
0 & -2 & 6 & -4
\end{array}
\right] \\
& R_3 \leftarrow R_3 + 2R_2 
\quad \longrightarrow 
\quad
\left[
\begin{array}{ccc|c}
1 & 2 & -1 & 1 \\
0 & 1 & -3 & 1 \\
0 & 0 & 0 & -2
\end{array}
\right]
\end{align*}

The last row corresponds to the equation \( 0 = -2 \), which is a contradiction. Therefore, the system is inconsistent and has \textbf{no solutions}.

\bigskip

(c) Augmented matrix:
\[
\left[
\begin{array}{cccc|c}
2 & 4 & 5 & 7 & 18 \\
1 & 2 & 1 & -1 & 3 \\
4 & 8 & 7 & 5 & 24
\end{array}
\right].
\]

Perform row operations to get the matrix in reduced row echelon form:

\begin{align*}
& R_1 \leftrightarrow R_2 
\quad \longrightarrow 
\quad
\left[
\begin{array}{cccc|c}
1 & 2 & 1 & -1 & 3 \\
2 & 4 & 5 & 7 & 18 \\
4 & 8 & 7 & 5 & 24
\end{array}
\right] \\
& R_2 \leftarrow R_2 - 2R_1 
\quad \longrightarrow
\quad
\left[
\begin{array}{cccc|c}
1 & 2 & 1 & -1 & 3 \\
0 & 0 & 3 & 9 & 12 \\
4 & 8 & 7 & 5 & 24
\end{array}
\right] \\
& R_3 \leftarrow R_3 - 4R_1 
\quad \longrightarrow
\quad
\left[
\begin{array}{cccc|c}
1 & 2 & 1 & -1 & 3 \\
0 & 0 & 3 & 9 & 12 \\
0 & 0 & 3 & 9 & 12
\end{array}
\right] \\
& R_3 \leftarrow R_3 - R_2 
\quad \longrightarrow
\quad
\left[
\begin{array}{cccc|c}
1 & 2 & 1 & -1 & 3 \\
0 & 0 & 3 & 9 & 12 \\
0 & 0 & 0 & 0 & 0
\end{array}
\right] \\
& R_2 \leftarrow \frac{1}{3} R_2 
\quad \longrightarrow
\quad
\left[
\begin{array}{cccc|c}
1 & 2 & 1 & -1 & 3 \\
0 & 0 & 1 & 3 & 4 \\
0 & 0 & 0 & 0 & 0
\end{array}
\right] \\
& R_1 \leftarrow R_1 - R_2
\quad \longrightarrow
\quad
\left[
\begin{array}{cccc|c}
1 & 2 & 0 & -4 & -1 \\
0 & 0 & 1 & 3 & 4 \\
0 & 0 & 0 & 0 & 0
\end{array}
\right]
\end{align*}

Hence, we have the system:

\[
\begin{cases}
x_1 + 2x_2 - 4x_4 = -1 \\
x_3 + 3x_4 = 4
\end{cases}
\]

We have 4 variables: \(x_1,x_2,x_3,x_4\) in total, with the pivot variables being \(x_1\) and \(x_3\), while the free variables being \(x_2\) and \(x_4\). \\

Let 
\[
x_2 = s,
\quad
x_4 = t
\]

Then from the second row: \(x_3 = 4 - 3t\).  
From the first row: \(x_1 = -1 - 2s + 4t\).

Hence the general solution is
\[
(x_1, x_2, x_3, x_4)
=
\bigl(-1 - 2s + 4t,\; s,\; 4 - 3t,\; t\bigr), \quad s,t \in \mathbb{R}
\]

\newpage

\section*{Exercise 2}
(a) Suppose $A$ is an $m \times n$ matrix with $m < n$. Consider $A$ as the coefficient matrix of a linear system, where the rightmost column of the augmented matrix is a zero column. Recall from class that we call the solution space of this system, the null space of $A$, which we denote by $\mathcal{N}(A)$. Show that the null space $\mathcal{N}(A)$ contains a nonzero vector by an argument involving the reduced row echelon form of $A$. \\

\noindent
(b) Use part (a) to prove that any $j$ vectors in $\mathbb{R}^k$ are linearly dependent if $j > k$. \\

\textbf{Solution:}
\newpage

\section*{Exercise 3}
(a) Suppose $S$ is a subset of a vector space $V$. Show that if $v \in V$ is contained in span$(S)$, then span$(S)$ = span$(S \cup \{v\})$. \\

\noindent
(b) Consider $V = \mathbb{R}^2$ and $S = \{(x,y) \mid x \geq 0 \text{ and } y \geq x\}$. Use part (a) to give a short proof that span$(S) = \mathbb{R}^2$ by showing that span$(S)$ contains the vectors $(1,0)$ and $(0,1)$. \\

\textbf{Solution:}
\newpage

\section*{Exercise 4}
Let $\mathbf{u}$ and $\mathbf{v}$ be distinct vectors in a vector space $V$. Show that $\{\mathbf{u}, \mathbf{v}\}$ is linearly dependent if and only if one of $\mathbf{u}$ or $\mathbf{v}$ is a scalar multiple of the other. \\

\textbf{Solution:}
\newpage

\section*{Exercise 5}
Either prove or give a counterexample to the following statement: If $\mathbf{v}_1, \mathbf{v}_2, \mathbf{v}_3$ are three vectors in $\mathbb{R}^3$ none of which is a scalar multiple of another, then they are linearly independent. \\

\textbf{Solution:}
\newpage

\section*{Exercise 6}
In the vector space $\mathcal{F}(\mathbb{R},\mathbb{R})$, all functions from $\mathbb{R}$ to $\mathbb{R}$, consider the elements $f(t) = \sin(t)$ and $g(t) = \cos(t)$. Is the subset $\{f,g\}$ linearly dependent or linearly independent? Prove your answer. \\

\textbf{Solution:}

\end{document}