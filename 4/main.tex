\documentclass{article}
\usepackage{graphicx}
\usepackage{amsmath}
\usepackage{array}
\usepackage{fancyhdr}
\usepackage{amssymb}
\usepackage[shortlabels]{enumitem}

\DeclareMathOperator{\R}{\mathbb R}

\pagestyle{fancy}
\fancyhead[L]{Banghao Chi}
\fancyhead[C]{Homework 4}
\fancyhead[R]{4th Mar}

\fancyfoot[C]{\thepage}

\renewcommand{\headrulewidth}{0.5pt}
\renewcommand{\footrulewidth}{0.5pt}

\begin{document}

\section*{Exercise 1}
For the given functions, prove that $T$ is a linear transformation, and find bases for both $\mathcal{N}(T)$ and $\mathcal{R}(T)$. Then compute the nullity and rank of $T$, and verify the dimension theorem. Finally, use the appropriate theorems in this section to determine whether $T$ is one-to-one or onto.
\begin{itemize}
    \item[(a)] $T : \mathbb{R}^3 \to \mathbb{R}^2$ defined by $T(a_1, a_2, a_3) = (a_1 - a_2, 2a_3)$.

    \item[(b)] $T : \mathbb{R}^2 \to \mathbb{R}^3$ defined by $T(a_1, a_2) = (a_1 + a_2, 0, 2a_1 - a_2)$.

    \item[(c)] $T : \mathcal{P}_2(\mathbb{R}) \to \mathcal{P}_3(\mathbb{R})$ defined by $T(f(x)) = xf(x) + f'(x)$.
\end{itemize}

\textbf{Solution: }\\

(a) $T : \mathbb{R}^3 \to \mathbb{R}^2$ defined by $T(a_1, a_2, a_3) = (a_1 - a_2, 2a_3)$ \\

First, we prove that $T$ is a linear transformation. \\

Let $\vec{u} = (u_1, u_2, u_3)$ and $\vec{v} = (v_1, v_2, v_3)$ be vectors in $\mathbb{R}^3$, and let $c$ be a scalar. \\

(i)
\begin{align*}
    T(\vec{u} + \vec{v}) &= T(u_1 + v_1, u_2 + v_2, u_3 + v_3) \\
    &= ((u_1 + v_1) - (u_2 + v_2), 2(u_3 + v_3)) \\
    &= ((u_1 - u_2) + (v_1 - v_2), 2(u_3 + v_3)) \\
    &= (u_1 - u_2, 2u_3) + (v_1 - v_2, 2v_3) \\
    &= T(\vec{u}) + T(\vec{v})
\end{align*}

(ii)
\begin{align*}
    T(c\vec{u}) &= T(cu_1, cu_2, cu_3) \\
    &= (cu_1 - cu_2, 2cu_3) \\
    &= c(u_1 - u_2, 2u_3) \\
    &= cT(\vec{u})
\end{align*}

Since both conditions are satisfied, $T$ is a linear transformation. \\

Next, we find a basis for $\mathcal{N}(T)$. \\

We want to find all vectors $(a_1, a_2, a_3)$ such that $T(a_1, a_2, a_3) = (0, 0)$.
This gives us the system:
\begin{align*}
a_1 - a_2 &= 0\\
2a_3 &= 0
\end{align*}

Solving, we get $a_1 = a_2$ and $a_3 = 0$. The general solution is $(a_1, a_1, 0)$ where $a_1$ is free.
Thus, a basis for $\mathcal{N}(T)$ is $\{(1, 1, 0)\}$. \\

Next, we find a basis for $\mathcal{R}(T)$. \\

We can determine the range by examining the images of the standard basis vectors:
\begin{align*}
T(1,0,0) &= (1,0)\\
T(0,1,0) &= (-1,0)\\
T(0,0,1) &= (0,2)
\end{align*}

Note that $(1,0) = -1(-1,0)$, so a basis for $\mathcal{R}(T)$ is $\{(1,0), (0,2)/2\} = \{(1,0), (0,1)\}$. \\

Next, we compute nullity and rank:
\begin{align*}
    \text{nullity}(T) &= \dim(\mathcal{N}(T)) = 1\\
    \text{rank}(T) &= \dim(\mathcal{R}(T)) = 2
\end{align*}

Next, we verify the dimension theorem:
\begin{align*}
    \text{nullity}(T) + \text{rank}(T) = 1 + 2 = 3 = \dim(\mathbb{R}^3)
\end{align*}

The dimension theorem is verified. \\

Finally, we determine if $T$ is one-to-one or onto:
\begin{itemize}
    \item Since $\text{nullity}(T) > 0$, $T$ is not one-to-one(since there exists a non-zero vector $v$ in the null space such that $T(v) = 0$, and therefore for any $u$ in the domain, $T(u) = T(u + v)$, mapping different inputs to the same output, making it not injective).
    \item Since $\text{rank}(T) = \dim(\mathbb{R}^2) = 2$, $T$ is onto(since the range of $T$ is the entire codomain $\mathbb{R}^2$, meaning every element of $\mathbb{R}^2$ is mapped to by some element of $\mathbb{R}^3$).
\end{itemize}

(b) $T : \mathbb{R}^2 \to \mathbb{R}^3$ defined by $T(a_1, a_2) = (a_1 + a_2, 0, 2a_1 - a_2)$ \\

First, we prove that $T$ is a linear transformation. \\

Let $\vec{u} = (u_1, u_2)$ and $\vec{v} = (v_1, v_2)$ be vectors in $\mathbb{R}^2$, and let $c$ be a scalar. \\

(i)
\begin{align*}
    T(\vec{u} + \vec{v}) &= T(u_1 + v_1, u_2 + v_2) \\
    &= ((u_1 + v_1) + (u_2 + v_2), 0, 2(u_1 + v_1) - (u_2 + v_2)) \\
    &= ((u_1 + u_2) + (v_1 + v_2), 0, (2u_1 - u_2) + (2v_1 - v_2)) \\
    &= (u_1 + u_2, 0, 2u_1 - u_2) + (v_1 + v_2, 0, 2v_1 - v_2) \\
    &= T(\vec{u}) + T(\vec{v})
\end{align*}

(ii)
\begin{align*}
    T(c\vec{u}) &= T(cu_1, cu_2) \\
    &= (cu_1 + cu_2, 0, 2cu_1 - cu_2) \\
    &= c(u_1 + u_2, 0, 2u_1 - u_2) \\
    &= cT(\vec{u})
\end{align*}

Since both conditions are satisfied, $T$ is a linear transformation. \\

Next, we find a basis for $\mathcal{N}(T)$. \\

We want to find all vectors $(a_1, a_2)$ such that $T(a_1, a_2) = (0, 0, 0)$.
This gives us the system:
\begin{align*}
a_1 + a_2 &= 0\\
2a_1 - a_2 &= 0
\end{align*}

From the first equation, $a_2 = -a_1$. Substituting into the second equation:
\begin{align*}
2a_1 - (-a_1) &= 0\\
2a_1 + a_1 &= 0\\
3a_1 &= 0
\end{align*}

Thus, $a_1 = 0$ and consequently $a_2 = 0$.
Therefore, $\mathcal{N}(T) = \{(0,0)\}$, and the basis is the empty set $\{\}$. \\

Next, we find a basis for $\mathcal{R}(T)$. \\

We can determine the range by examining the images of the standard basis vectors:
\begin{align*}
T(1,0) &= (1,0,2)\\
T(0,1) &= (1,0,-1)
\end{align*}

These two vectors are linearly independent, so a basis for $\mathcal{R}(T)$ is $\{(1,0,2), (1,0,-1)\}$. \\

Next, we compute nullity and rank:
\begin{align*}
    \text{nullity}(T) &= \dim(\mathcal{N}(T)) = 0\\
    \text{rank}(T) &= \dim(\mathcal{R}(T)) = 2
\end{align*}

Next, we verify the dimension theorem:
\begin{align*}
    \text{nullity}(T) + \text{rank}(T) = 0 + 2 = 2 = \dim(\mathbb{R}^2)
\end{align*}

The dimension theorem is verified. \\

Finally, we determine if $T$ is one-to-one or onto:
\begin{itemize}
    \item Since $\text{nullity}(T) = 0$, $T$ is one-to-one (since the only vector that maps to zero is the zero vector itself, meaning different inputs map to different outputs).
    \item Since $\text{rank}(T) = 2 < 3 = \dim(\mathbb{R}^3)$, $T$ is not onto (since the range of $T$ is a proper subspace of the codomain $\mathbb{R}^3$, meaning there are elements in $\mathbb{R}^3$ that cannot be reached as the image of any vector in $\mathbb{R}^2$).
\end{itemize}

(c) $T : \mathcal{P}_2(\mathbb{R}) \to \mathcal{P}_3(\mathbb{R})$ defined by $T(f(x)) = xf(x) + f'(x)$ \\

First, we prove that $T$ is a linear transformation. \\

Let $f(x)$ and $g(x)$ be polynomials in $\mathcal{P}_2(\mathbb{R})$, and let $c$ be a scalar. \\

(i)
\begin{align*}
    T(f(x) + g(x)) &= x(f(x) + g(x)) + (f(x) + g(x))' \\
    &= xf(x) + xg(x) + f'(x) + g'(x) \\
    &= (xf(x) + f'(x)) + (xg(x) + g'(x)) \\
    &= T(f(x)) + T(g(x))
\end{align*}

(ii)
\begin{align*}
    T(c \cdot f(x)) &= x(c \cdot f(x)) + (c \cdot f(x))' \\
    &= c \cdot xf(x) + c \cdot f'(x) \\
    &= c \cdot (xf(x) + f'(x)) \\
    &= c \cdot T(f(x))
\end{align*}

Since both conditions are satisfied, $T$ is a linear transformation. \\

Next, we find a basis for $\mathcal{N}(T)$. \\

Let $f(x) = a_0 + a_1x + a_2x^2 \in \mathcal{P}_2(\mathbb{R})$. Then:
\begin{align*}
    T(f(x)) &= x(a_0 + a_1x + a_2x^2) + (a_1 + 2a_2x) \\
    &= a_0x + a_1x^2 + a_2x^3 + a_1 + 2a_2x \\
    &= a_1 + (a_0 + 2a_2)x + a_1x^2 + a_2x^3
\end{align*}

For $T(f(x)) = 0$, all coefficients must be 0:
\begin{align*}
    a_1 &= 0 \\
    a_0 + 2a_2 &= 0 \\
    a_2 &= 0
\end{align*}

With $a_2 = 0$ and $a_0 + 2a_2 = 0$, we get $a_0 = 0$.
Therefore, $a_0 = a_1 = a_2 = 0$, so $\mathcal{N}(T) = \{0\}$, and the basis is the empty set $\{\}$. \\

Next, we find a basis for $\mathcal{R}(T)$. \\

We can determine the range by examining the images of the standard basis for $\mathcal{P}_2(\mathbb{R})$:
\begin{align*}
    T(1) &= x \cdot 1 + (1)' = x \\
    T(x) &= x \cdot x + (x)' = x^2 + 1 \\
    T(x^2) &= x \cdot x^2 + (x^2)' = x^3 + 2x
\end{align*}

These polynomials are linearly independent, so a basis for $\mathcal{R}(T)$ is $\{x, x^2 + 1, x^3 + 2x\}$. \\

Next, we compute nullity and rank:
\begin{align*}
    \text{nullity}(T) &= \dim(\mathcal{N}(T)) = 0 \\
    \text{rank}(T) &= \dim(\mathcal{R}(T)) = 3
\end{align*}

Next, we verify the dimension theorem:
\begin{align*}
    \text{nullity}(T) + \text{rank}(T) = 0 + 3 = 3 = \dim(\mathcal{P}_2(\mathbb{R}))
\end{align*}

The dimension theorem is verified. \\

Finally, we determine if $T$ is one-to-one or onto:
\begin{itemize}
    \item Since $\text{nullity}(T) = 0$, $T$ is one-to-one (since the null space contains only the zero polynomial, meaning different polynomials map to different outputs).
    \item Since $\text{rank}(T) = 3 < 4 = \dim(\mathcal{P}_3(\mathbb{R}))$, $T$ is not onto (since the range of $T$ is a proper subspace of the codomain $\mathcal{P}_3(\mathbb{R})$, meaning there are polynomials in $\mathcal{P}_3(\mathbb{R})$ that cannot be reached as the image of any polynomial in $\mathcal{P}_2(\mathbb{R})$).
\end{itemize}

\newpage

\section*{Exercise 2}
\begin{itemize}
    \item[(a)] Suppose that $T : \mathbb{R}^2 \to \mathbb{R}^2$ is linear, $T(1,0) = (1,4)$ and $T(1,1) = (2,5)$. What is $T(2,3)$? Is $T$ one-to-one?

    \item[(b)] Give an example of a linear transformation $T : \mathbb{R}^2 \to \mathbb{R}^2$ such that $\mathcal{N}(T) = \mathcal{R}(T)$.
\end{itemize}

\textbf{Solution: }\\

(a) Since $T$ is linear, we have $T(1,1) = T(1,0) + T(0,1)$ \\

This gives us:
\begin{align*}
(2,5) &= (1,4) + T(0,1)\\
\end{align*}

Solving for $T(0,1)$:
\begin{align*}
T(0,1) &= (2,5) - (1,4) = (1,1)\\
\end{align*}

Now we find $T(2,3)$ using linearity:
\begin{align*}
T(2,3) &= 2T(1,0) + 3T(0,1)\\
&= 2(1,4) + 3(1,1)\\
&= (2,8) + (3,3)\\
&= (5,11)\\
\end{align*}

To determine if $T$ is one-to-one, we want to show that $T(v) = 0$ implies $v = 0$. \\

Suppose $T(a,b) = (0,0)$ for some vector $(a,b)$. Using linearity:
\begin{align*}
T(a,b) &= aT(1,0) + bT(0,1)\\
&= a(1,4) + b(1,1)\\
&= (a+b, 4a+b)\\
&= (0,0)
\end{align*}

Solving, we get $a = 0$ and $b = 0$, so $(a,b) = (0,0)$, which proves that $T$ is one-to-one. \\

(b) We need to find a linear transformation $T$ such that $\mathcal{N}(T) = \mathcal{R}(T)$ \\

Consider the transformation defined by the matrix:
\begin{align*}
A &= \begin{bmatrix} 0 & 1 \\ 0 & 0 \end{bmatrix}\\
\end{align*}

For any vector $(x,y)$, we have:
\begin{align*}
A\begin{bmatrix} x \\ y \end{bmatrix} &= \begin{bmatrix} y \\ 0 \end{bmatrix}\\
\end{align*}

The null space consists of vectors $(x,y)$ such that $A(x,y) = (0,0)$, so:
\begin{align*}
\mathcal{N}(A) &= \{(x,0) : x \in \mathbb{R}\}\\
\end{align*}

The range consists of all possible outputs of $A$:
\begin{align*}
\mathcal{R}(A) &= \{(z,0) : z \in \mathbb{R}\}\\
\end{align*}

Both are the $x$-axis, so $\mathcal{N}(A) = \mathcal{R}(A)$ \\

Therefore, the transformation $T(x,y) = (y,0)$ satisfies the requirement.

\newpage

\section*{Exercise 3}
Let $V,W$ be vector spaces, with $\dim(V) = n, \dim(W) = m$, and $n > m$.

\begin{itemize}
    \item[(a)] Show that there is no one-to-one linear transformation $T : V \to W$.

    \item[(b)] Show that there is no onto linear transformation $T : W \to V$ (notice that $V,W$ have flipped in this expression!)

    \item[(c)] Show that a linear map $T : V \to W$ need not be onto by giving an example where it is not.
\end{itemize}

\textbf{Solution: }\\

(a) Show that there is no one-to-one linear transformation $T : V \to W$. \\

Suppose there exists a one-to-one linear transformation $T : V \to W$. \\

Since $T$ is one-to-one, we know that $\dim(\mathcal{N}(T)) = 0$. \\

By the Rank-Nullity theorem:
\begin{align*}
\dim(V) &= \dim(\mathcal{N}(T)) + \dim(\mathcal{R}(T))\\
n &= 0 + \dim(\mathcal{R}(T))\\
\end{align*}

So $\dim(\mathcal{R}(T)) = n$. But since $\mathcal{R}(T) \subseteq W$, we must have $\dim(\mathcal{R}(T)) \leq \dim(W) = m$, which means $n \leq m$, contradiction. \\

Therefore, there cannot be a one-to-one linear transformation $T : V \to W$. \\

(b) Show that there is no onto linear transformation $T : W \to V$. \\

Suppose there exists an onto linear transformation $T : W \to V$. \\

Since $T$ is onto, $\dim(\mathcal{R}(T)) = \dim(V) = n$. \\

By the Rank-Nullity theorem:
\begin{align*}
\dim(W) &= \dim(\mathcal{N}(T)) + \dim(\mathcal{R}(T))\\
m &= \dim(\mathcal{N}(T)) + n
\end{align*}

Since $\dim(\mathcal{N}(T)) \geq 0$, we have $m \geq n$, which contradicts our given condition that $n > m$. \\

Therefore, there cannot be an onto linear transformation $T : W \to V$. \\

(c) Show that a linear map $T : V \to W$ need not be onto by giving an example where it is not. \\

Let $V = \mathbb{R}^3$ and $W = \mathbb{R}^2$, so $n = 3 > m = 2$. \\

Define $T: \mathbb{R}^3 \to \mathbb{R}^2$ by $T(x, y, z) = (0, y)$ for all $(x, y, z) \in \mathbb{R}^3$. \\

It is linear because:
\begin{align*}
T((x_1, y_1, z_1) + (x_2, y_2, z_2)) &= T(x_1 + x_2, y_1 + y_2, z_1 + z_2)\\
&= (0, y_1 + y_2)\\
&= (0, y_1) + (0, y_2)\\
&= T(x_1, y_1, z_1) + T(x_2, y_2, z_2)
\end{align*}

And for any scalar $c$:
\begin{align*}
T(c(x, y, z)) &= T(cx, cy, cz)\\
&= (0, cy)\\
&= c(0, y)\\
&= cT(x, y, z)
\end{align*}

The image of $T$ is $\{(0, y) : y \in \mathbb{R}\}$, which is just the $y$-axis of $\mathbb{R}^2$. Since there are points in $\mathbb{R}^2$ (like $(1, 0)$) that are not in the image of $T$, the transformation is not onto.

\newpage

\section*{Exercise 4}
Given bases $\beta$ and $\gamma$ of $\mathbb{R}^n$ and $\mathbb{R}^m$, respectively, for each linear transformation $T : \mathbb{R}^n \to \mathbb{R}^m$, compute $[T]_{\beta}^{\gamma}$.

\begin{itemize}
    \item[(a)] $T : \mathbb{R}^2 \to \mathbb{R}^3$ with $\beta, \gamma$ standard bases and $T(a_1, a_2) = (2a_1 - a_2, 3a_1 + 4a_2, a_1)$.

    \item[(b)] $T : \mathbb{R}^3 \to \mathbb{R}^2$ with $\beta, \gamma$ standard bases and $T(a_1, a_2, a_3) = (2a_1 + 3a_2 - a_3, a_1 + a_3)$.

    \item[(c)] $T : \mathbb{R}^2 \to \mathbb{R}^3$ with $\beta$ standard basis for $\mathbb{R}^2$, $\gamma = \{(1,1,0),(0,1,1),(2,2,3)\}$ and $T(a_1, a_2) = (a_1 - a_2, a_1, 2a_1 + a_2)$.

    \item[(d)] $T : \mathbb{R}^2 \to \mathbb{R}^3$ with $\beta = \{(1,2),(2,3)\}, \gamma = \{(1,1,0),(0,1,1),(2,2,3)\}$ and $T(a_1, a_2) = (a_1 - a_2, a_1, 2a_1 + a_2)$.
\end{itemize}

\textbf{Solution: }\\



\newpage

\section*{Exercise 5}
Let
\begin{align*}
\alpha = \left\{\left(\begin{matrix} 1 & 0 \\ 0 & 0 \end{matrix}\right), \left(\begin{matrix} 0 & 1 \\ 0 & 0 \end{matrix}\right), \left(\begin{matrix} 0 & 0 \\ 1 & 0 \end{matrix}\right), \left(\begin{matrix} 0 & 0 \\ 0 & 1 \end{matrix}\right)\right\}, \quad \beta = \{1, x, x^2\} \quad \text{and} \quad \gamma = \{1\}.
\end{align*}

\begin{itemize}
    \item[(a)] Define $T : \mathcal{M}_{2\times2}(\mathbb{R}) \to \mathcal{M}_{2\times2}(\mathbb{R})$ by $T(A) = A^t$. Compute $[T]_{\alpha}$.

    \item[(b)] Define 
    \begin{align*}
    T : \mathcal{P}_2(\mathbb{R}) \to \mathcal{M}_{2\times2}(\mathbb{R}) \text{ by } T(f) = \left(\begin{matrix} f'(0) & 2f(1) \\ 0 & f''(3) \end{matrix}\right),
    \end{align*}
    where $'$ denotes differentiation. Compute $[T]_{\beta}^{\alpha}$.

    \item[(c)] Define $T : \mathcal{M}_{2\times2}(\mathbb{R}) \to \mathbb{R}$ by $T(A) = \text{Tr}(A) = \text{sum of diagonal elements of }A$. Compute $[T]_{\alpha}^{\gamma}$.

    \item[(d)] Define $T : \mathcal{P}_2(\mathbb{R}) \to \mathbb{R}$ by $T(f(x)) = f(2)$. Compute $[T]_{\beta}^{\gamma}$.

    \item[(e)] If $A = \left(\begin{matrix} 1 & -2 \\ 0 & 4 \end{matrix}\right)$ compute $[A]_{\alpha}$.

    \item[(f)] If $f(x) = 3 - 6x + x^2$, compute $[f]_{\beta}$.

    \item[(g)] For $a \in \mathbb{R}$, compute $[a]_{\gamma}$.
\end{itemize}

\textbf{Solution: }\\



\newpage

\section*{Exercise 6}
We define the linear transformation $T_{\theta} : \mathbb{R}^2 \to \mathbb{R}^2$ to be rotation counter-clockwise about the origin through angle $\theta$. Let $T_x$ be the transformation that reflects in the $x$-axis.

\begin{itemize}
    \item[(a)] Write down the matrices of $T_{\theta}$ and $T_x$ with the respect to the standard basis $\beta = \{(1,0),(0,1)\}$ for $\mathbb{R}^2$.

    \item[(b)] Show that for $\theta \in (0,\pi) \cup (\pi,2\pi)$ one has
    \begin{align*}
    T_x \circ T_{\theta} \neq T_{\theta} \circ T_x.
    \end{align*}

    \item[(c)] Next, show that there is some angle $\psi$ such that
    \begin{align*}
    T_x \circ T_{\psi} = T_{\theta} \circ T_x.
    \end{align*}

    \item[(d)] What is the relationship between $\theta$ and $\psi$? Discuss the geometric meaning of this computation.
\end{itemize}

\textbf{Solution: }\\



\newpage

\end{document}